\begin{introduction}

L'id\'ee pour ce projet est n\'ee d'une discussion entre mon directeur de recherche, M. Guy Tremblay, et la professeure Magda Fusaro.
Mme Fusaro enseignait un cours o\`u les \'etudiants devraient rendre plusieurs petits travaux de r\'edaction sur Moodle.
La correction de ces nombreux travaux \'etait un travail fastidieux qui prenait de nombreuses heures.

Les \'etudiants devaient remplir un document Word et l'envoyer sur Moodle.
Or, la correction d'un texte dans un document Word o\`u tout est possible est tr\`es complexe, les \'etudiants peuvent facilement ne pas respecter le gabarit.
Nous avons donc convenu de r\'ealiser notre projet imm\'ediatement dans Moodle au lieu de faire un outil externe, car Moodle permet de faire un questionnaire en ligne qui est facilement balisable.

Moodle est une plateforme utilisée par plusieurs établissements scolaires au Québec, notamment, la plupart des cégeps et quelques universités l'offrent à leurs enseignants.
J'ai d'ailleurs eu à utiliser cette plateforme en tant qu'étudiant à l'Université du Québec à Montréal et aussi en tant qu'enseignant au Cégep régional de Lanaudière, à Joliette.

Moodle permet, entre autres, de créer et de corriger des travaux, quiz ou examens en ligne.
La correction est une tâche complexe et g\'en\'eralement peu appréciée par les enseignants et enseignantes.
L'idée générale de ce projet est donc d'aider les enseignants et correcteurs dans leur tâche de correction à l'intérieur d'un environnement Moodle.
Plus sp\'ecifiquement, le but du projet est de créer un module d'extension Moodle d'aide à la correction des textes à développement qui fera, entre autres, la mise en évidence de mots-clés fournis par l'enseignant.

Ce document pr\'esente les principales étapes de réalisation du projet.
Premièrement, nous avons d\^u en apprendre plus sur Moodle en tant que correcteur et en tant que programmeur afin de créer le bon type de module d'extension.
Deuxièmement, nous avons d\^u trouver les méthodes de détection de mots-clés et choisir celle qui semblait la plus appropri\'ee.
Troisièmement, nous avons d\^u comprendre comment tester notre module d'extension selon les bonnes pratiques de la plateforme Moodle.
Finalement, nous avons développ\'e et test\'e notre module d'extension.

\end{introduction}
