\begin{introduction}

Moodle est une plateforme utilisée par plusieurs établissements scolaires au Québec, notamment, la plupart des cégeps et quelques universités l'offrent à leurs enseignants.
J'ai d'ailleurs eu à utiliser cette plateforme en tant qu'étudiant à l'Université du Québec à Montréal et aussi en tant qu'enseignant au Cégep régional de Lanaudière, à Joliette.

Moodle permet, entre autres, de créer et de corriger des travaux, quiz ou examens en ligne.
La correction est une tâche complexe et g\'en\'eralement peu appréciée par les enseignants et enseignantes.
L'idée générale de ce projet est donc d'aider les enseignants et correcteurs dans leur tâche de correction à l'intérieur d'un environnement Moodle.
Plus sp\'ecifiquement, le but du projet est de créer un module d'extension Moodle d'aide à la correction des textes à développement qui fera, entre autres, la mise en évidence de mots-clés fournis par l'enseignant.


\GT{Peut-\^etre mentionner que la motivation initiale de ce projet est
venue de discussions entre ton directeur de recherche et la
professeure Magda Fusaro, qui avait enseign\'e un cours o\`u les
\'etudiant.e.s rendaient un grand nombre de petits travaux, sur
lesquelles elle avait d\^u consacrer de nombreuses heures \`a
corriger.  Apr\`es discussion entre ton directeur et elle, il avait
sembl\'e qu'un module tel que celui que tu d\'ecris lui aurait \'et\'e
utile.
%
Ensuite, dans les r\'esultats et limites --- voir le commentaire que je te
sugg\`ere d'ajouter --- tu pourras mentionner que Mme Fusaro \'etant
entre temps devenue Rectrice de l'UQAM, n'a pas pu collaborer \`a la
mise en place d'un essai r\'eel.
%
De cette fa\c{c}on, cela expliquera mieux la motivation initiale du
projet, et aussi certaines de ses limites.}


Ce document pr\'esente les principales étapes de réalisation du projet.
Premièrement, nous avons d\^u en apprendre plus sur Moodle en tant que correcteur et en tant que programmeur afin de créer le bon type de module d'extension.
Deuxièmement, nous avons d\^u trouver les méthodes de détection de mots-clés et choisir celle qui semblait la plus appropri\'ee.
Troisièmement, nous avons d\^u comprendre comment tester notre module d'extension selon les bonnes pratiques de la plateforme Moodle.
Finalement, nous avons développ\'e et test\'e notre module d'extension.

\end{introduction}
