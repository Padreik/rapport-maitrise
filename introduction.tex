\begin{introduction}

Moodle est une plateforme utilisée par plusieurs établissements scolaires au Québec, la plupart des cégeps et quelques universités l'offrent à leurs enseignants.
J'ai d'ailleurs eu à utiliser cette plateforme en tant qu'étudiant à l'Université du Québec à Montréal ainsi qu'en tant qu'enseignant au cégep régional de Lanaudière à Joliette.

Moodle permet, entre autres, de créer et de corriger des examens en ligne.
La correction est une tâche complexe et peu appréciée dans le domaine de l'enseignement.
L'idée générale de ce projet est donc d'aider les enseignants et correcteurs dans leur tâche de correction à l'intérieur d'un environnement Moodle.
Le but du projet est donc de créer un module d'extension Moodle d'aide à la correction des textes à développement qui fera, entre autres, la mise en évidence de mots-clés donnés par l'enseignant.

Ce document suit les étapes de réalisation du projet.
Premièrement, en apprendre plus sur Moodle en tant que correcteur et en tant que programmeur afin de créer le bon type de module d'extension.
Deuxièmement, trouver les méthodes de détections de mots-clés et choisir la meilleure.
Troisièmement, trouver comment tester un module d'extension Moodle selon les bonnes pratiques de la plateforme.
Finalement, développer et tester le module d'extension.

\end{introduction}