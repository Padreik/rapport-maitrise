\chapter{R\'esultats et limites}
L'affichage de la r\'eponse de l'enseignant et de l'\'etudiant c\^ote \`a c\^ote en plus des instructions au correcteur devraient aider le correcteur dans sa t\^ache de correction.
De plus, la possibilit\'e de choisir la langue de racination permet aux enseignants d'utiliser l'aide \`a la correction dans la langue de leur choix.
Un probl\`eme a \'et\'e d\'etect\'e vers la fin du projet: la s\'eparation du texte en mots peut poser probl\`eme.
On s\'epare chaque mot en utilisant tout symbole non alphanum\'erique et les caract\`eres d'espacements.
Donc le mot \og aujourd'hui \fg{} est consid\'er\'e comme deux mots, car l'apostrophe est un symbole non alphanum\'erique.
Par contre, \og qu'autrui \fg{} doit \^etre consid\'er\'e comme deux mots diff\'erents, ce qui est correct dans notre cas.
De plus, comme le module d'extension supporte plusieurs langues, chaque langue comporte ses particularit\'es.
Par exemple en anglais le mot \og \textit{wasn't} \fg{} est une compression des mots \og \textit{was} \fg{} et \og \textit{not} \fg{} ce qui ne sera pas bien g\'er\'e.
Ce probl\`eme n'est pas assez majeur pour nuire \`a la majorit\'e des situations.
M\^eme si \og aujourd'hui \fg{} est consid\'er\'e comme deux mots, la mise en \'evidence se fera sur chacun de ces deux mots.
En anglais, les mots probl\'ematiques (comme \og \textit{wasn't} \fg{}) seront rarement utilis\'es en tant que mots-cl\'es.
Comme ce probl\`eme n'est pas majeur et que la solution serait tr\`es complexe (remplacer la racination par la lemmatisation ou faire de l'analyse syntaxique), nous avons d\'ecidé d'ignorer ce probl\`eme.
\`A la suite de l'analyse du probl\`eme pr\'ec\'edent, nous avons d\'ecouvert un deuxi\`eme probl\`eme.
Pour mettre en \'evidence les mots, on se base sur un dictionnaire qui associe le mot \`a la racine.
Si le mot-cl\'e est \og aimer \fg{} et le texte contient les mots \og aimera \fg{} et \og aimerait \fg{}, les deux mots seront mis en \'evidence.
Par contre, la partie du mot \og aimera \fg{} du mot \og aimerait \fg{} sera mise en \'evidence deux fois, car il contient l'autre mot qui doit \^etre mis en \'evidence.
La mise en \'evidence fait un simple remplacement de cha\^ine du mot par ce m\^eme mot entour\'e de balises \texttt{HTML} comme le d\'emontre l'exemple de code \ref{code:mots-cles}.
Une mise en \'evidence double ne sera pas diff\'erente visuellement d'une mise en \'evidence simple, car une balise de style \texttt{HTML} ou \texttt{CSS} dans une balise du m\^eme style ne fera pas de modification suppl\'ementaire.
En utilisant l'exemple pr\'ec\'edent, bien que \og aimerait \fg{} serait en partie mis en \'evidence deux fois, le mot ne sera soulign\'e qu'une seule fois.
Plusieurs petites am\'eliorations pourraient \^etre ajout\'ees au module d'extension:
La mise en \'evidence des mot-cl\'es est encore simple, elle pourrait \^etre configurable via l'interface d'administration afin de changer le style de mise en \'evidence.
En ce moment la mise en \'evidence met le mot en gras et soulign\'e.
La racination utilis\'ee est aussi limit\'ee~: le correcteur doit lire le contexte afin de confirmer si le mot-cl\'e est valide, probl\`eme qui pourrait \^etre \'elimin\'e avec un outil de lemmatisation.
Finalement, les mots-cl\'es en soi ne sont pas affich\'es au correcteur, ce qui pourrait \^etre un ajout int\'eressant.
Dans la planification du projet, il \'etait pr\'evu de pouvoir comparer les textes des \'etudiants entre eux (au lieu de seulement comparer \`a l'enseignant).
Dans un premier temps nous voulions simplement afficher les r\'eponses des autres \'etudiants \`a la place de la r\'eponse de l'enseignant afin de comparer les r\'eponses semblables.
Par contre, cette id\'ee demandait une analyse syntaxique afin de trouver les r\'eponses semblables, ce qui sortait du cadre du projet.
Cet ajout pourrait \^etre un projet int\'eressant pour un futur \'etudiant.
Toutes ces limitations et correctifs \`a apporter mettent en doute la compl\'etude du module d'extension d\'evelopp\'e.
Il ne sera donc pas soumis au r\'epertoire des modules d'extensions Moodle.
Apr\`es am\'eliorations, il pourrait toutefois devenir un ajout int\'eressant pour les enseignants et correcteurs du monde entier.
Le code source est disponible sur GitHub comme le stipule les r\`egles de d\'eveloppement Moodle.
Il se trouve \`a l'adresse \url{https://github.com/Padreik/moodle-qtype_essayhelper}.
