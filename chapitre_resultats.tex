\chapter{R\'esultats et limites}

L'affichage de la réponse de l'enseignant et de l'étudiant côte à côte en plus des instructions au correcteur devraient aider le correcteur dans sa tâche de correction.
De plus, la possibilit\'e de choisir la langue de racination permet aux enseignants d'utiliser l'aide \`a la correction dans la langue de leur choix.

Un probl\`eme a \'et\'e d\'etect\'e vers la fin du projet.
La s\'eparation du texte en mots peut causer probl\`eme.
On s\'epare chaque mot en utilisant tout symbole non-alphanum\'erique et les caract\`eres d'espacements.
Donc le mot \og aujourd'hui \fg{} est consid\'er\'e comme deux mots car l'apostrophe est un symbole non-alphanum\'erique.
Par contre, \og qu'autruis \fg{} doit \^etre consid\'er\'e comme deux mots diff\'erents.
De plus, comme le module d'extension supporte plusieurs langues, chaque langue comporte ses particularit\'es.
Par exemple en anglais le mot \og wasn't \fg{} est une compression des mots \og was \fg{} et \og not \fg{} ce qui ne sera pas bien g\'er\'e.
Ce probl\`eme n'est pas assez majeur pour nuire \`a la majorit\'e des situations.
M\^eme si \og aujourd'hui \fg{} est consid\'er\'e comme deux mots, la mise en \'evidence se fera sur chacun de ces deux mots.

Un probl\`eme technique \`a \'et\'e d\'ecouvert vers la fin du projet.
Pour mettre en \'evidence les mots, on se base sur un dictionnaire qui associe le mot \`a la racine.
Si un mot-cl\'e est \og aimer \fg{} et le texte contient les mots \og aimera \fg{} et \og aimerait \fg{} les deux mots seront mis en \'evidence.
Par contre, la partie du mot \og aimera \fg{} du mot \og aimerait \fg{} sera mis en \'evidence deux fois car il contient l'autre mot qui doit \^etre mis en \'evidence (la mise en \'evidence fait un simple remplacement de cha\^ine du mot par ce m\^eme mot entour\'e de balises \texttt{HTML} comme le d\'emontre l'exemple de code \ref{code:mots-cles}).

Plusieurs petites am\'eliorations pourraient \^etre ajout\'es au module d'extension:
La mise en évidence des mots clés est encore simple, elle pourrait être configurable via l'interface d'administration afin de changer le style de mise en évidence.
La racination utilisée est aussi limitée~: le correcteur doit lire le contexte afin de confirmer si le mot-clé est valide, problème qui pourrait être éliminé avec un outil de lemmatisation.
Finalement, les mots-clés en soit ne sont pas affichés au correcteur, ce qui pourrait être un ajout intéressant.

Dans la planification du projet, il était prévu de pouvoir comparer les textes des étudiants entre eux (au lieu de seulement comparer à l'enseignant).
Par contre, cette idée demandait une analyse syntaxique qui sortait du cadre du projet.
Cet ajout pourrait être un projet intéressant pour un futur étudiant.

Toutes ces limitations et correctifs à apporter mettent en doute la complétude du module d'extension développé.
Il ne sera donc pas soumis au répertoire des modules d'extensions Moodle.
Après améliorations, il pourrait toutefois devenir un ajout intéressant pour les enseignants et correcteurs du monde entier.

Le code source est disponible sur GitHub comme le stipule les règles de développement Moodle.
Il se trouve à l'adresse \url{https://github.com/Padreik/moodle-qtype_essayhelper}.