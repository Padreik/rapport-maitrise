\documentclass[12pt]{memoireuqam1.3}

\usepackage{graphicx}% Pour les figures
\usepackage[french]{babel}
\usepackage[utf8]{inputenc} % Pour utiliser les caractères accentués
\usepackage[T1]{fontenc}
\usepackage{hyperref}
\usepackage{listings}
\usepackage{float}
\usepackage{caption}

\usepackage{color}

\definecolor{Gray}{RGB}{245,245,245}
\definecolor{CPPGray}{RGB}{120,120,120}

\lstdefinestyle{default}{
  showspaces=false,
  showtabs=false,
  breaklines=true,
  showstringspaces=false,
  breakatwhitespace=true,
  basicstyle=\ttfamily,
  backgroundcolor=\color{Gray},
  numbers=left, numbersep=3pt, stepnumber=1, numberstyle=\tiny\color{CPPGray},
  rulecolor=\color{CPPGray},
  basicstyle=\footnotesize
}

\definecolor{dkblue}{rgb}{0,0,.6}

\lstset{language=php,
  keywordstyle = \color{dkblue},
  style=default}

\newfloat{lstfloat}{h}{lstfloat}
\floatname{lstfloat}{Code}

\newcommand\listcodename{Liste des exemples de code}
\newcommand\listofcodes{% Pris de memoireuqam1.3.cls
{
    \listof{lstfloat}{\listcodename}
    \addtocontents{toc}{\protect\vspace{1.5ex}}
    \addcontentsline{toc}{chapter}{\UpperRef{\listcodename}}
    }
}

%\input macro  % Pour les définitions personnelles

\newcommand{\GT}[1]{{\footnotesize\textcolor{red}{[[(GT) #1]]}}}


\begin{document}

%%%%%%%%%%%%%%%%%%%%
% Pour la page titre
%%%%%%%%%%%%%%%%%%%%
\title{Développement d'un module d'extension Moodle d'aide à la correction de questions de type \og texte~long \fg{}}
% Votre nom complet tel qu'il apparaît à votre dossier du registrariat de l'UQAM
\author{Philippe Girard}
% Année et mois courant sauf si spécifié autrement pas \degreemonth et \degreeyear
%\degreemonth{mois du dépôt}
%\degreeyear{année du dépôt}
\uqamprojet %\uqammemoire %% ou \uqamthese ou \uqamrapport
\matiere{Génie logiciel}


\thispagestyle{empty}        % La page titre n'est pas numérotée
\maketitle

%%%%%%%%%%%%%%%%%%%%
% Page préliminaires
%%%%%%%%%%%%%%%%%%%%
% \renewcommand \bibname{R\'EF\'ERENCES}% FACULTATIF
%si vous voulez qu'apparaisse le titre RÉFÉRENCES plutôt que BIBLIOGRAPHIE

\renewcommand \listfigurename{LISTE DES FIGURES}
\renewcommand \appendixname{APPENDICE}
\renewcommand \figurename{Figure}
\renewcommand \tablename{Tableau}

\pagenumbering{roman} % numérotation des pages en chiffres romains
\addtocounter{page}{1} % Pour que les remerciements commencent à la page ii
\chapter*{Remerciements}

Premi\`erement, j'aimerais remercier mon directeur de ma\^itrise, M. Guy Tremblay.
Sans son support, ce projet n'aurait pas \'et\'e un succ\`es.
Je tiens aussi \`a souligner sa disponibilit\'e pour la tenue de nos rencontres \`a l'ext\'erieur de l'UQAM.

Merci aussi au C\'egep R\'egional de Lanaudi\`ere, \`a Joliette, qui m'a aid\'e au niveau financier dans ce projet.

J'aimerais aussi remercier mes parents et amis, qui ont su croire en ce projet malgr\'e une date de fin toujours repouss\'ee.
Mention sp\'eciale \`a Marc-Andr\'e, qui m'a aid\'e \`a rester motiv\'e tout au long du projet.

\tableofcontents % Pour générer la table des matières
\listoftables % Pour générer la liste des tableaux
\listoffigures % Pour générer la liste des figures
\listofcodes % Pour générer la liste des exemples de code
\begin{abstract}

Ce projet a consist\'e à développer un module d'extension (\textit{plugin}) Moodle pour aider la correction de textes à développement dans une activité de type questionnaire.

Le module d'extension développé est de type \og Type de question \fg{} basé sur le module d'extension \texttt{qtype\_essay}.
Le nouveau type de question se comporte  comme le type \texttt{qtype\_essay} hormis qu'il ne permet pas la remise de fichiers ou l'utilisation du \textit{WYSIWYG (What You See Is What You Get)}.
Plus sp\'ecifiquement, ce nouveau module ajoute les fonctionnalités suivantes au module de correction manuelle: mise en évidence de mots-clés trouvés par racination (\textit{stemming}) et affichage de la réponse de l'enseignant \`a cot\'e de celle de l'étudiant, pour en faciliter la comparaison.

Des tests unitaires ainsi que des tests d'acceptation ont été écrits afin de valider le fonctionnement de ce module d'extension.

Les fonctionnalités de ce module d'extension sont limit\'ees. 
Notamment, des ajouts tels que les suivants seraient utiles pour apporter une aide plus importante au correcteur: remplacer la racination par la lemmatisation afin d'améliorer l'identification des mots-clés~; comparer les textes des étudiants entre eux afin de détecter le plagiat~; effectuer une analyse syntaxique afin de donner une estimation de la note que mérite un texte basé sur les réponses précédemment corrigées.

MOTS-CLÉS: Moodle; Aide \`a la correction; Racination; Tests unitaires et d'acceptation.

\end{abstract}

% Utilisez l'environnement  abstract pour rédiger votre résumé


%%%%%%%%%%%%%%%%%%%%
% Document principal
%%%%%%%%%%%%%%%%%%%%

\begin{introduction}

Introduction...

\end{introduction}
% Utilisez l'environnement  introduction pour rédiger votre introduction
\chapter{Introduction \`a Moodle}
Moodle est un environnement d'apprentissage en ligne (LMS, \textit{Learning Management System}) cr\'e\'e par Martin Dougiamas.
Cette plateforme est un logiciel libre d\'evelopp\'e en PHP sous \href{http://docs.moodle.org/dev/License}{licence GPL (\textit{GNU Public License})} dont le code source se trouve sur \href{https://github.com/moodle/moodle}{GitHub}.
Moodle est un outil modulaire o\`u chaque enseignant peut cr\'eer ses cours comme il le souhaite.
Chaque cours est s\'epar\'e en sections d\'efinies en semaine, module, th\`eme ou selon plusieurs autres configurations possibles.
Chaque section comporte plusieurs activit\'es: un texte \`a lire, des documents \`a t\'el\'echarger, un forum pour \'echanger, un sondage ou questionnaire en ligne \`a compl\'eter, un devoir \`a remettre, et plus encore.
Chaque activit\'e est hautement configurable.
Par exemple, un devoir peut \^etre fait individuellement ou en \'equipe, avoir une date et une heure de d\'ebut et de fin de disponibilit\'e avec un d\'elai de retard permis, etc.
Dans les questionnaires en ligne, un enseignant peut cr\'eer une banque de questions et s\'electionner les questions d\'esir\'ees ou laisser le syst\`eme choisir les questions al\'eatoirement.
\section{Activit\'e questionnaire}
Ce projet s'attarde \`a l'activit\'e Questionnaire de Moodle, plus sp\'ecifiquement sur la question de type Texte long (parfois appel\'e Composition, selon la version de Moodle).
Tout d'abord, qu'est-ce qu'une activit\'e Questionnaire?
Une telle activit\'e permet \`a l'enseignant de cr\'eer un formulaire en ligne auquel les \'etudiants devront r\'epondre.
Il peut s'agir, par exemple, d'une \'evaluation sommative qui sera cumul\'ee au carnet de notes Moodle, d'une \'evaluation formative ou d'un atelier qui pourra \^etre refait plusieurs fois jusqu'\`a ce que l'\'etudiant atteigne une note pr\'ed\'efinie.
Parmi les options les plus utiles de cette activit\'e, on retrouve (toutes optionnelles): d\'ebut et fin de disponibilit\'e du questionnaire, temps disponible \`a partir de l'ouverture du questionnaire, note de passage, ordre al\'eatoire des questions, nombre de tentatives possibles, etc.
Chaque questionnaire est compos\'e d'un nombre de questions choisies dans une banque de questions.
Les questions peuvent avoir un ordre pr\'ecis ou \^etre choisies al\'eatoirement parmi un ensemble de questions.
Voici une liste non exhaustive des types de questions:
\begin{description}
  \item[Choix multiple]
  
  G\'en\`ere une liste de boutons radio ou de cases \`a cocher;
  
  \item[R\'eponse courte]
  
  Affiche un champ texte pouvant accueillir quelques mots;
  
  \item[Num\'erique]
  
  Affiche un champ texte pour valeur num\'erique pouvant prendre en compte une unit\'e de mesure (km, cm ou m par exemple);
  
  \item[Question Cloze]
  
  Permets de cr\'eer un texte lacunaire, chaque \og trou \fg{} pouvant \^etre rempli avec une sous-question de type choix multiple, r\'eponse courte ou num\'erique;
  
  \item[Composition]
  
  Affiche un champ texte multiligne, multiligne avec police monospace ou WYSIWYG (\textit{What You See Is What You Get}).
  L'\'etudiant doit \'ecrire un texte \`a d\'eveloppement.
\end{description}
Toute question, except\'e celles de type Composition, peut se corriger automatiquement \`a un moment d\'ecid\'e par l'enseignant.
La correction peut se faire \`a la saisie de la r\'eponse pour une r\'etroaction en direct ou apr\`es la soumission du questionnaire.
Par exemple, dans un examen \`a choix de r\'eponses, les \'etudiants pourraient voir leur note d\`es qu'ils ont termin\'e leur examen.
Chaque type de question se corrige de mani\`ere un peu diff\'erente:
\begin{description}
  \item[Choix multiple]
  
  Chaque bouton radio ou case \`a cocher vaut un certain nombre de points, allant de -100\% \`a 100\% de la valeur de la question.
  Par exemple, dans une question valant 2 points avec plusieurs choix possibles, chacune des deux bonnes r\'eponses ajoute 1 point et chacune des mauvaises r\'eponses enl\`eve 1 point.
  Dans cet exemple, l'\'etudiant pourrait avoir une note n\'egative;
  
  \item[R\'eponse courte]
  
  L'enseignant peut donner plusieurs mots ou expressions r\'eguli\`eres qui valent un certain nombre de points.
  \`a la question \og{} \`a quelle ann\'ee Christophe Colomb a-t-il d\'ecouvert l'Am\'erique? \fg{}, une r\'eponse contenant \og 1492 \fg{} pourrait donner tous les points alors qu'une r\'eponse correspondant \`a l'expression r\'eguli\`ere \og 15(.*)si\`ecle \fg{} pourrait donner la moiti\'e des points;
  
  \item[Num\'erique]
  
  L'enseignant peut donner plusieurs bonnes r\'eponses en tenant compte de plusieurs unit\'es.
  Par exemple les r\'eponses \og 2 km \fg{} et  \og 2000 m \fg{} pourraient \^etre accept\'es;
  
  \item[Question Cloze]
  
  Chaque \og trou \fg{} du texte lacunaire est corrig\'e selon le type de question utilis\'e dans ce \og trou \fg{}.
  Si on utilise une question de type R\'eponse courte dans un \og trou \fg{}, Moodle utilisera la correction correspondante \`a ce type de question.
  
  \item[Composition]
  
  Moodle ne corrige pas automatiquement ce type de question.
  L'enseignant ou le correcteur doit donc corriger manuellement chaque copie.
\end{description}
\section{Question de type Composition}
\GT{Partie plus bas pas claire~: Tu dis qu'il faut corriger
manuellement, mais tu dis que Moodle fournit un module de correction
manuelle. Reformuler!  Du genre, fournit un module qui aide \`a
effectuer la correction manuelle, qui aide le correcteur \`a traiter
l'ensemble des r\'eponses, ou quelque chose du genre!?}
\PG{J'ai ajout\'e plus de d\'etail sur la correction automatique dans le chapitre pr\'ec\'edent et enlev\'e une phrase douteuse dans le chapitre qui suit.}

D\`u \`a sa nature, le type de question Composition est le seul qui ne peut pas se corriger automatiquement par Moodle.
L'enseignant doit donc corriger manuellement toutes les r\'eponses \`a l'aide du module de correction manuelle tel qu'illustr\'e \`a la figure \ref{correction-manuelle}.
\begin{figure}[b!]
  \includegraphics[scale=0.6]{images/correction-manuelle.png}
  \caption{Interface du module de correction manuelle de Moodle.}
  \label{correction-manuelle}
\end{figure}
L'enseignant peut choisir de corriger le questionnaire en entier, un \'etudiant apr\`es l'autre.
Il peut aussi corriger une question \`a la fois, affichant ainsi les r\'eponses de tous les \'etudiants \`a cette question sur une seule page.
Cette derni\`ere m\'ethode est appr\'eci\'ee de plusieurs enseignants, car elle permet de comparer toutes les r\'eponses donn\'ees et ainsi corriger de mani\`ere plus uniforme.
Notre projet consiste donc \`a faciliter le travail du correcteur en cr\'eant un nouveau module Moodle qui:
\begin{enumerate}
  \item Affichera la r\'eponse de l'\'etudiant et la r\'eponse de l'enseignant c\^ote \`a c\^ote afin de faciliter la t\^ache au correcteur;
  \item Surlignera les mots cl\'es, fournis par l'enseignant, dans la r\'eponse de l'\'etudiant et de l'enseignant;
  \item Utilisera la racination ou la lemmisation afin de surligner les mots cl\'es, peu importe leur accord.
\end{enumerate}
Le module d'extension propos\'e sera donc utile pour les questions \`a d\'eveloppement pour les textes descriptifs, et aussi pour des segments de code (cours  de programmation), mais le sera moins pour les textes d'opinion o\`u il n'y a pas de \og bonne \fg{} r\'eponse.
\section{Les modules d'extensions Moodle}
\GT{Soit tu donnes, dans une note en bas de page, ce qui signifie
SCORM, soit tu laisses tomber compl\`etement cet acronyme.}
\PG{J'ai laiss\'e tomb\'e l'acronyme, \c{c}a n'a pas vraiment d'importance dans le cadre du projet.}
L'aspect modulaire de Moodle existe tant pour les enseignants que pour les programmeurs.
Presque toutes les fonctionnalit\'es de Moodle sont configur\'ees \`a l'aide de modules d'extensions.
Il y a des modules d'extensions pour changer l'apparence du site, pour exporter les donn\'ees, pour changer le fonctionnement des questions, etc.
Chaque module d'extension est d\'evelopp\'e par un membre de la communaut\'e.
La plupart de ces modules d'extension se retrouvent sur GitHub et quelques-uns se retrouvent sur le r\'epertoire de module d'extensions Moodle\footnote{\url{https://moodle.org/plugins/}}.
Pour ajouter un nouveau module d'extension \`a ce r\'epertoire, il faut qu'il soit approuv\'e par un comit\'e Moodle.
Le processus d'approbation est pr\'esent\'e \`a la figure~\ref{plugin-workflow}.
Il est possible d'installer un module d'extension non officiel en installant les fichiers au bon endroit dans l'arborescence Moodle.
\begin{figure}[h!]
  \includegraphics[scale=0.7]{images/plugin-contribution-workflow.png}
  \caption[Flux d'acception d'un module d'extension Moodle]{Flux d'acception d'un module d'extension Moodle (\href{https://docs.moodle.org/dev/Plugin_contribution}{\url{https://docs.moodle.org/dev/Plugin\_contribution}}).}
  \label{plugin-workflow}
\end{figure}
Pour ajouter des fonctionnalit\'es \`a l'activit\'e questionnaires, il y a trois types de modules d'extensions \`a consid\'erer: un rapport de questionnaire, un type de question et un comportement de question.
\subsection{Module d'extension rapport de questionnaire (\textit{Quiz report})}
Un rapport de questionnaire sert principalement \`a corriger les r\'eponses des \'etudiants.
On peut aussi se servir de ce type de module d'extension pour afficher les r\'eponses et les notes des \'etudiants pour un questionnaire sous forme de rapport.
Finalement, on peut aussi s'en servir afin de modifier les champs disponibles dans la configuration de base d'un questionnaire.
Lors de l'analyse initiale du projet, il avait \'et\'e pr\'evu de se servir uniquement de ce type de module d'extension, car nous voulons uniquement modifier l'interface de correction.
Par contre, comme nous voulions ajouter une liste de mots cl\'es \`a une question et que ce type de module d'extension permet de modifier seulement le questionnaire, d\'evelopper uniquement un module d'extension de ce type n'\'etait pas suffisant.
\subsection{Module d'extension type de question (\textit{Question type})}
Chaque question est d\'efinie par un type de question (voir la liste \`a la section~1.1).
Un nouveau type de question ajoute donc plus de choix \`a l'enseignant d\'esirant cr\'eer un questionnaire.
Chaque type de question poss\`ede des champs personnalis\'es et change l'apparence de la question pour l'\'etudiant et l'enseignant.
On ne voulait pas modifier le module d'extension \og Type de question Composition \fg{} car les nouvelles fonctionnalit\'es ne seront pas utiles pour tous les types de texte: un texte d'opinion pour un cours de philosophie, par exemple, n'aura pas n\'ecessairement besoin d'une liste de mots-cl\'es et d'une fonctionnalit\'e de comparaison de textes.
La possibilit\'e de cr\'eer un module d'extension qui ajoute un nouveau type de question qui \'etend les fonctionnalit\'es du \og Type de question Composition \fg{} par h\'eritage a \'et\'e analys\'ee, mais il semble que ce n'est pas possible avec Moodle.
Seule alternative restante: se baser sur le \og Type de question Composition \fg{} pour cr\'eer un tout nouveau module d'extension.
Ce nouveau module d'extension utilise les m\^emes fonctionnalit\'es que le \og Type de question Composition \fg{}, mais en enlevant quelques fonctionnalit\'es (remise de pi\`ece jointe et \'ecriture dans un WYSIWIG) et en ajoutant quelques autres (mots-cl\'es et r\'eponse de l'enseignant).
\subsection{Module d'extension comportement de question (\textit{Question behaviour})}
Un comportement de question permet les op\'erations suivantes:
\begin{enumerate}
  \item Ajouter du code \`a la suite de la question, par exemple, un bouton \og V\'erifier \fg{} ou une zone de texte commentaire;
  
  \item Modifier la m\'ethode de correction, par exemple, une correction manuelle, une correction automatique instantan\'ee quand l'\'etudiant r\'epond ou une correction automatique lorsque le test est termin\'e;
  
  \item Modifier l'affichage des questions, par exemple, afficher une question suppl\'ementaire si l'\'etudiant a plusieurs fautes ou donner des indices \`a l'\'etudiant sous r\'eserve d'une perte de points.
\end{enumerate}
Comme nous d\'esirons une correction manuelle, nous pouvons utiliser le comportement \og ManualGraded \fg{}.
De plus, comme il est possible de contr\^oler directement l'apparence de la r\'eponse et de la question avec un module d'extension type de question, il n'est pas n\'ecessaire de cr\'eer un module d'extension \og Comportement de question \fg{}.
\section{Choix des modules d'extensions \`a cr\'eer}
Un module d'extension type de question est obligatoire dans ce projet afin d'ajouter les champs mots-cl\'es et r\'eponse de l'enseignant.
Nous n'avons pas besoin des fonctionnalit\'es qu'offre le module d'extension comportement de question.
Il reste \`a d\'eterminer s'il faut utiliser un module d'extension rapport de questionnaire ou non.
En analysant le code du module d'extension de correction manuelle fournie de base avec Moodle (\texttt{quiz\_grading}), on d\'ecouvre que ce module fournit l'interface de correction, mais que la r\'eponse de chaque \'etudiant est g\'er\'ee par le type de question dans un affichage en lecture seulement.
Les fonctionnalit\'es pr\'evues peuvent donc se trouver autant dans le type de question que dans un rapport de questionnaire, la complexit\'e du code est la m\^eme pour les deux cas.
Puisqu'il doit y avoir obligatoirement un module d'extension type de question et qu'un nouveau module d'extension rapport de questionnaire est facultatif, nous avons d\'ecid\'e de ne faire qu'un seul module d'extension.
En outre, le fait d'avoir un seul module d'extension facilite grandement l'installation pour d'autres administrateurs Moodle.
\chapter{Détection des mots-clés}
\label{chap:keywords}

Afin de détecter les mots-clés, nous avons décidé d'utiliser la racine de chaque mot.
En utilisant la racine de chaque mot, nous pouvons trouver les mots-clés, peu importe leur accord ou le temps de verbe.
Deux techniques de recherche de racine d'un mot ont été retenus: la lemmatisation et la racination (\textit{stemming}).

Lors du choix de la technique utilisée, nous avons considéré que:

\begin{enumerate}
  \item Le but du projet n'étant pas de faire de l'analyse linguistique, une librairie devra être utilisée si possible;
  \item La librairie doit être écrite en PHP, langage utilisé par Moodle, afin de faciliter l'installation du module d'extension développé;
  \item La librairie doit permettre de trouver la racine des mots en français, en anglais et plusieurs autres langues si possible afin de permettre l'utilisation du module d'extension à travers le monde.
\end{enumerate}

\section{La lemmatisation}

La lemmatisation consiste à trouver le lemme de chaque mot.
Le lemme est la forme canonique d'un mot, soit l'infinitif pour un verbe, le masculin singulier pour un adjectif, etc.
Par exemple, le lemme du verbe \og aimerait \fg{} est \og aimer \fg{}.
Cette technique utilise habituellement un dictionnaire qui associe toutes les conjugaisons possibles d'un mot à leur lemme.
Certaines librairies vont même considérer le contexte afin de trouver le bon lemme.

Les librairies de lemmatisation écrites en PHP sont très rares.
Lors de nos recherches, aucune de ces librairies ne supportait le français et l'anglais.

\section{La racination}

La racination consiste à trouver la racine (parfois appelé radical ou stemme) de chaque mot.
La racine est trouvée par algorithme en enlevant la fin du mot.
Par exemple, la racine de \og aimerait \fg{} est \og aim \fg{}.
Cette technique ignore la plupart des exceptions.

La racination est moins précise que la lemmatisation.
Par exemple la racine de \og courons \fg{} est la même que la racine de \og couronnement \fg{}, soit \og couron \fg{}.
La lemmatisation ne fera pas la même erreur et trouvera les lemmes \og courir \fg{} et \og couronne \fg{}.
En contrepartie, la recherche de la racine sera plus rapide avec un algorithme de racination qu'une recherche dans un dictionnaire de lemmes.

Il existe quelques librairies de racination en PHP.
Une d'elles est \href{https://github.com/wamania/php-stemmer}{php-stemmer} qui fait la racination en français, anglais, et 10 autres langues.
Les algorithmes utilisés par php-stemmer sont basés sur les algorithmes écrits en Snowball, un langage développé pour la racination.
L'algorithme de racination en anglais ainsi que Snowball ont été développés par Martin Porter.

Comme nous avons trouvé une librairie de racination qui satisfait tous les critères et aucune librairie de lemmatisation n'en fait autant, ce projet va utiliser php-stemmer pour la découverte de mots-clés.

\section{Snowball}

Snowball est un petit langage de traitement de texte pour la racination de texte.
Une application Snowball peut être compilée en C ou en Java.
Le site web de Snowball donne l'algorithme de racination pour plusieurs langues: anglais, français, espagnol, allemand, russe, etc.
L'algorithme pour la langue anglaise se base sur l'algorithme de Porter.
L'auteur a créé une deuxième version en se basant sur les algorithmes des langues latines.
L'origine des algorithmes donnés pour les langues latines, ce qui inclut la langue française, n'est pas donnée, mais semble venir aussi de Porter avec l'aide de contributeurs.

La libraire php-stemmer a été écrite en PHP par Wamania en se basant sur les algorithmes écrits en Snowball.
Le reste de cette section tente d'expliquer la logique derrière l'algorithme de Snowball pour le français.
L'algorithme est séparé en six étapes en plus d'une étape de préparation et d'une étape finale.

\subsection*{Préparation}

Premièrement, le mot est transformé en minuscule par php-stemmer.
Les lettres u et i précédées et suivies par une voyelle, les lettres y précédées ou suivies par une voyelle ainsi que les u précédées de q sont mises en majuscules.
Les lettres en majuscules sont considérées en tant que consonnes dans l'algorithme.
Par exemple, dans le mot \og joUer \fg{}, le U ne sera pas considéré comme une minuscule.
Pour déterminer si une lettre est une voyelle, l'algorithme utilise une liste de lettres (a, e, i, o, u, y, â, à, ë, é, ê, è, ï, î, ô, û et ù).
Donc une voyelle en majuscule ne fera pas partie de la liste.

Ensuite il faut trouver trois sections dans le mot: RV (right vowel), R1 (right 1) et R2 (right 2).
Dans les exemples qui suivent, les lettres soulignées indiquent les lettres faisant partie du groupe.

\begin{description}
  \item[RV]
  
  Si le mot débute par deux voyelles, RV débute après la troisième lettre (oub\underline{lier}).
  Si le mot commence par \og par, col ou tap \fg{}, RV définit les lettres qui suivent (col\underline{onne}).
  Dans les autres cas, RV débute après la première voyelle excluant la première lettre (bo\underline{njour}, algo\underline{rithme}).
  Si aucune de ses règles ne s'applique, RV est vide (un sigle comme IBN).
  RV n'existe pas dans l'algorithme de la langue anglaise.
  
  \item[R1]
  
  R1 débute après la première consonne précédée d'une voyelle (aim\underline{er}, tap\underline{is}).
  R1 est vide s'il n'y a pas de telle consonne.
  
  \item[R2]
  
  R2 débute après la première consonne précédée d'une voyelle dans R1 (fameus\underline{ement}).
  R2 est vide s'il n'y a pas de telle consonne.
\end{description}

RV peut contenir R1 et inversement.
Les lettres qui ne sont pas dans R1 ni dans RV ne seront pas touchées sauf pour quelques exceptions.

\subsection*{Étape 1: Suffixes standards}

Cette étape sert à trouver la racine des mots, adjectifs, adverbes, etc.

Trouver le suffixe le plus long parmi une liste donnée et faire l'action correspondante.
Voici des exemples de la liste qui compte 43 suffixes et 24 règles.

\begin{itemize}
  \item \textbf{ance iqUe isme able ...}: Supprimer si dans R2;
  \item \textbf{atrice ateur ation ...}: Supprimer si dans R2. Si précédé de \textbf{ic}, supprimer si dans R2, sinon remplacer par \textbf{iqU};
  \item \textbf{logie logies}: Remplacer par \textbf{log} si dans R2;
  \item \textbf{eaux}: Remplacer par \textbf{eau};
  \item \textbf{aux}: Remplacer par \textbf{al} si dans R1;
  \item \textbf{amment}: Remplacer par \textbf{ant} si dans RV;
  \item \textbf{emment}: Remplacer par \textbf{ent} si dans RV;
  \item \textbf{ment ments}: Supprimer si dans RV.
\end{itemize}

\subsection*{Étape 2: Suffixes de verbes}

Cette étape trouve la racine d'un verbe.
On cherche et enlève les suffixes standards.

Faire cette étape seulement si l'étape 1 n'a rien changé ou si un des suffixes suivants a été trouvé: amment, emment, ment et ments.

\subsubsection*{Étape 2a: Suffixes débutant par i}

Trouver le suffixe le plus long parmi la liste ci-dessous.
S'il est précédé d'une consonne et que le tout est dans RV, supprimer le suffixe excluant la consonne supplémentaire.

îmes, ît, îtes, i, ie, ies, ir, ira, irai, iraIent etc.

\subsubsection*{Étape 2b: Autres suffixes de verbes}

Trouver le suffixe le plus long parmi les trois listes ci-dessous.
Les listes sont non-exaustives afin d'alléger le texte.

\begin{description}
  \item[ions]
  
  Supprimer si dans RV et R2;
  
  \item[é ée ées és èrent er era erai eraIent ...]
  
  Supprimer si dans RV;
  
  \item[âmes ât âtes a ai aIent ...]
  
  Supprimer si dans RV. Si la lettre e précède le suffixe et qu'elle se trouve aussi dans RV, la supprimer.
\end{description}

\subsection*{Étape 3: Dernière lettre}

Si les étapes 1 ou 2 ont modifié le mot, faire cette étape, sinon passer à l'étape 4.

Si la dernière lettre est un Y, la remplacer par i.
Si c'est un ç, la remplacer par c.

\subsection*{Étape 4: Résidue du suffixe}

Faire cette étape seulement si les étapes 1 à 3 n'ont pas modifié le mot.
Cette étape sert à enlever le pluriel et le féminin des mots proches de leurs racines.

Si un mot se termine par \textbf{s} et n'est pas précédé de \textbf{a, i, o, u, è ou s}, supprimer ce s final.

Si un mot se termine par \textbf{ion}, que cette finale se trouve dans RV et dans R2 et qu'elle est précédée de la lettre \textbf{s ou t} (cette dernière doit se trouver dans RV), supprimer la finale \textbf{ion}.

Si un mot se termine par \textbf{ier, ière, Ier ou Ière} et que cette finale se trouve dans RV, remplacer cette dernière par \textbf{i}.

Si un mot se termine par \textbf{e} et que cette lettre ce trouve dans RV, la supprimer.

Si un mot se termine par \textbf{ë}, que cette lettre se trouve dans RV et qu'elle est précédée de \textbf{qu}, la supprimer.

\subsection*{Étape 5: Dédoubler la lettre finale}

Les suppressions et remplacements des étapes précédentes peuvent laisser une faute dans la racine du mot.
Cette étape sert à enlever les lettres doubles de certaines finales de mots.

Si un mot se termine par \textbf{enn, onn, ett, ell ou eill}, supprimer la dernière lettre.

\subsection*{Étape 6: Accent final}

Cette étape aussi sert a nettoyer la racine du mot.

Si un mot se termine par la lettre \textbf{è ou é} suivit d'une ou de plusieurs consonnes, enlever l'accent de ce \textbf{e}.

\subsection*{Finalisation}

Finalement, on enlève les majuscules sur les voyelles ajoutées durant l'étape de préparation.

Au final, on obtient la racine du mot.
Lors de la comparaison de texte, il peut y avoir quelques problèmes.
Par exemple les mots \og acceptables \fg{} et \og accepté \fg{} ont la même racine: \og accept \fg{}.
Par contre le sens des 2 mots est différent.
\chapter{Tests avec Moodle}

Moodle offre deux types de tests: des tests d'acceptation et des tests unitaires.

L'environnement utilisé afin de rouler les tests est une machine virtuelle Xubuntu~16.04 mise à jour en date du 14 décembre 2017.
La version de PHP est 7.0.22.
Le moteur de base de données est MySQL version~5.7.20 installé avec les paquets Ubuntu.
La base de données utilise la collation \textit{utf8mb4\_unicode\_ci} comme conseillée dans \href{https://docs.moodle.org/34/en/MySQL}{la documentation}.
La version de Moodle est la dernière version stable à ce jour, soit la version~3.4.
Moodle a été installé avec git à partir de la branche \textit{MOODLE\_34\_STABLE}, \textit{commit} \href{https://github.com/moodle/moodle/commit/a45c46600021667691dbb4bce5420a2f65d3239c}{a45c466} déployé le 14 décembre 2017.
Les tests ont été exécutés une seule fois avant le développement du module d'extension afin de confirmer le fonctionnement de l'environnement.

\section{Tests unitaires}

Les tests unitaires effectuent des vérifications à petite échelle.
Chaque fonction dans le code est isolée et testée.
Pour isoler une fonction, il faut remplacer les dépendances par des \textit{mocks}.
Un \textit{mock} va simuler la dépendance utilisée par la fonction en retournant une valeur fixe.
De cette manière, il est possible de tester cette fonction uniquement sans être impacté par les autres fonctions.
Plusieurs tests peuvent être effectués afin de valider tous les cas possibles.

Une installation vanille de Moodle vient avec plusieurs tests unitaires.
Le code de base ainsi que les modules d'extensions de base sont déjà testés.
Ces tests fonctionnent avec \textit{PHPUnit}, un \textit{framework} de tests unitaires pour PHP.

\begin{lstfloat}
\begin{lstlisting}[frame=l]
class qtype_essay_question_test extends advanced_testcase {
    public function test_get_question_summary() {
        $essay = test_question_maker::make_an_essay_question();
        $essay->questiontext = 'Hello <img src="http://example.com/globe.png" alt="world" />';
        $this->assertEquals('Hello [world]', $essay->get_question_summary());
    }
\end{lstlisting}
\caption{Exemple de test unitaire du module d\'extension \textit{qtype\_essay}.}
\label{code:unittest}
\end{lstfloat}

Avant de déployer le module d'extension, la suite de tests \textit{PHPUnit} a été exécutée.
Il y a 8750 tests et 90~575 vérifications (\textit{assertions}) à exécuter.
8682 tests ont réussi (\textit{success}), 67 tests ont été ignorés (\textit{skipped}) et un test a échoué (\textit{failures}).
Les tests ignorés sont, par exemple, les tests pour LDAP et les tests pour Redis, deux technologies qui n'ont pas été configurées dans notre cas.

Il y avait malheureusement une erreur causée par l'encodage des caractères dans la base de données MySQL.
Le test vérifie que la base de données fonctionne en UTF8 et considère la casse (\textit{case-sensitive}) lors de la comparaison de texte.
La version de la base de données MySQL utilisée pour les tests est 5.7.
Or, il n'y a pas d'encodage UTF8 sensible à la casse pour les versions précédentes à MySQL~5.8.
L'erreur est connue et détaillé \href{https://docs.moodle.org/dev/Database_collation_issue}{sur le site de référence Moodle}.

Cette erreur peut poser problème dans les questions de type numérique ainsi que pour les remises de fichiers.
Par exemple, on ne peut pas entrer les unités \og km \fg{} et \og Km \fg{} dans les réponses possibles, car la base de données considère les unités identiques.
Par contre, la correction de la question, exécutée en PHP, fait la différence entre les deux unités.
Un étudiant qui réponds avec l'unité \og 10 KM \fg{} alors que l'enseignant a enregistré la réponse \og 10 km \fg{}, aura une erreur.
Comme le module d'extension développé n'utilisera pas la comparaison de texte à partir de la base de données, le développement peut continuer sans problème.

Si notre application est entièrement testée avec des tests unitaires, nous sommes assurés qu'il ne devrait pas y avoir de problème de code dans nos fonctions.
Si on compare les tests à une équipe sportive, les tests unitaires analysent chaque joueur, mais ne considèrent pas le travail d'équipe.
Il faut donc ajouter un autre type de test, les tests d'acceptation, afin de s'assurer de l'efficacité du travail d'équipe.

\section{Tests d'acceptation}

Un test d'acceptation valide que l'application correspond aux exigences du logiciel.
Ce type de test peut s'exécuter à partir de l'interface, de la même manière qu'un humain pourrait tester le logiciel.
Moodle vient aussi avec plusieurs de ces tests, mais ce ne sont pas tous les modules d'extensions qui en ont.

Les tests d'acceptation sont exécutés à l'aide de Behat, un \og framework \fg{} PHP d'automatisation de tests qui se base sur le \og \textit{Behavior Driven Development} (BDD) \fg{}.
Un serveur ou une application Selenium exécutera les tests en ligne de commande ou dans un navigateur web, selon le besoin.
Les tests sont écrits en anglais, compréhensibles par tous.
Chaque instruction et vérification doit être, préalablement, configurée en PHP.

La structure des tests d'acceptation avec behat est structuré comme suis:

\begin{itemize}
  \item La première ligne avec les \@ permet de catégoriser chaque fichier de test.
        Ça permet d'exécuter les tests d'acceptations pour un seul module d'extension ou pour un type de module d'extension;
        
  \item Débutant par \textit{Feature}, le titre du test afin de le retrouver facilement;
  
  \item Les 3 lignes suivantes permettent de décrire le test au lecteur.
        Moodle les écrit habituellement comme suit:
        
        \begin{itemize}
          \item \og \textit{As a ...} \fg{} décrit quel type d'utilisateur est ciblé par ce test;
          \item \og \textit{In order to ...} \fg{} décrit l'action à tester;
          \item \og \textit{I need to ...} \fg{} décrit ce qu'il faut vérifier.
        \end{itemize}
        
  \item Ensuite on débute le \textit{Background} qui prépare le test.
        La première action de cette section débutera par \textit{Given} et toutes les autres par \textit{And}.
        Chaque action prépare l'environnement pour le test, par exemple: ajouter un enregistrement dans la base de données, naviguer à une certaine page, cliquer sur un bouton, etc;
        
  \item Ensuite il y a une ou plusieurs sections \textit{Scenario} qui définie chaque test à effectuer.
        Sur la même ligne que le \textit{Scenario} il y a une description du test pour le lecteur.
        Ensuite, le \textit{When} qui définit l'action à tester.
        Finalement, le \textit{Then} qui définit le comportement entendu suite au test.
\end{itemize}

\begin{lstfloat}
\begin{lstlisting}[frame=l]
@qtype @qtype_essay
Feature: Test creating an Essay question
  As a teacher
  In order to test my students
  I need to be able to create an Essay question

  Background:
    Given the following "users" exist:
      | username | firstname | lastname | email               |
      | teacher1 | T1        | Teacher1 | teacher1@moodle.com |
    And the following "courses" exist:
      | fullname | shortname | category |
      | Course 1 | C1        | 0        |
    And the following "course enrolments" exist:
      | user     | course | role           |
      | teacher1 | C1     | editingteacher |
    And I log in as "teacher1"
    And I follow "Course 1"
    And I navigate to "Question bank" node in "Course administration"

  Scenario: Create an Essay question with Response format set to 'HTML editor'
    When I add a "Essay" question filling the form with:
      | Question name            | essay-001                      |
      | Question text            | Write an essay with 500 words. |
      | General feedback         | This is general feedback       |
      | Response format          | HTML editor                    |
    Then I should see "essay-001"
\end{lstlisting}
\caption{Test d'acceptation du module d\'extension \textit{qtype\_essay}.}
\label{code:behattest}
\end{lstfloat}

La série de tests d'acceptation de Moodle a été exécutée avant le développement du module d'extension.
Il y a un total de 1771 scénarios et un total de 43~824 étapes.
4 scénarios et 102 étapes ont été ignorés et 6 étapes et 6 scénarios ont échoués.

Voici une description des erreurs ainsi que leurs influences sur le développement dans ce projet:

\begin{itemize}
  \item Erreur lorsqu'un étudiant passe d'une activité à une autre.
        Notre module d'extension se concentre sur une seule activité, ce cas n'est donc pas problématique.
        
  \item Erreur dans le filtre du calendrier mensuel.
        Notre module d'extension ne touche pas au calendrier, ce cas n'est donc pas problématique.
        
  \item Erreur dans la navigation entre les modes de groupes.
        Notre module d'extension ne touche pas aux modes de groupes, ce cas n'est donc pas problématique.
        
  \item Solr (engin de recherche) n'est pas installé sur l'environnement de test.
        Notre module d'extension ne touche pas à la recherche, ce cas n'est donc pas problématique.
        
  \item Erreur Solr identique à la précédente.
  
  \item Erreur dans la liste des étudiants, l'enseignant ne voit pas quels étudiants sont actifs.
        Notre module d'extension ne touche pas à la liste des étudiants, ce cas n'est donc pas problématique.
\end{itemize}

Comme les 6 cas ne sont pas problématiques, le développement peut se poursuivre sans problème.
Lors de l'exécution finale des tests, il ne devrait y avoir que ces six mêmes erreurs.
\chapter{Développement}

Le développement d'une extension Moodle n'est pas intuitif.
La documentation n'est pas toujours complète et l'interaction dans le code entre les types d'extensions rapport de questionnaire, type de question et comportement de question n'est pas toujours claire.
Dans le développement de cette extension, le code de l'extension \og qtype\_essay \fg{} a été utilisé comme documentation.
En prenant cette extension comme base, il a été facile de construire ce qui était désiré.

Les commentaires de copyrights ont été ajustés comme suit:

\begin{lstlisting}[frame=l]
/**
 * Essay for correction helper question definition class.
 *
 * @package    qtype
 * @subpackage essayhelper
 * @copyright  2017 Philippe Girard
 * @license    http://www.gnu.org/copyleft/gpl.html GNU GPL v3 or later
 *
 * Inspired by:
 * @package    qtype
 * @subpackage essay
 * @copyright  2009 The Open University
 * @license    http://www.gnu.org/copyleft/gpl.html GNU GPL v3 or later
 */
\end{lstlisting}

\section{Fonctionnalités}

Plusieurs fonctionnalités de l'extension \og qtype\_essay \fg{} ont été enlevées:

\begin{description}
  \item[Éditeur de texte WYSIWIG]
  
  Il permet de modifier l'apparence du texte facilement.
  Ça permet, entre autres, de surligner, de souligner, de mettre en gras, de mettre en italique et plus encore.
  Comme on ne peut pas enlever des options WYSIWIG pour une seule extension, ça devient complexe de trouver une mise en forme qui permettra de mettre de l'emphase sur les mots clés, car un étudiant pourrait, par erreur, reproduire la même mise en forme.
  
  \item[Remise de fichier]
  
  Il est possible d'écrire directement dans la zone de texte ou de remettre un document texte (configurable par l'enseignant).
  Comme l'extension n'aidera pas à corriger les textes remis par fichier, cette option a été enlevée.
\end{description}

Plusieurs fonctionnalités sont restées dans la nouvelle extension:

\begin{description}
  \item[Rétroaction générale]
  
  Permets de laisser un commentaire à l'étudiant une fois que la correction est disponible.
  Par exemple, il pourrait laisser la réponse officielle selon les notes de cours ou des explications pour les erreurs courantes.
  
  \item[Modèle de réponse]
  
  Préremplis la zone de texte de l'étudiant avec le texte donné.
  Par exemple, un en-tête de fonction pour une question de programmation ou la liste des mots à définir.
  
  \item[Information de l'évaluateur]
  
  Affiche un texte pour le correcteur seulement.
  Utile pour voir facilement le barème de correction ou donner des instructions au correcteur.
  Est affiché sous la réponse de l'étudiant lors de la correction.
\end{description}

Finalement, les fonctionnalités suivantes ont été ajoutées:

\begin{description}
  \item[Mots-clés]
  
  Les mots-clés seront mis en évidence dans la réponse de l'étudiant lors de la correction manuelle

  \item[Réponse officielle de l'enseignant]
  
  Affiche ce texte à droite de la réponse de l'étudiant lors de la correction.
  Les mots-clés seront mis en évidence aussi dans ce texte.
\end{description}

\section{Détection des mots clés}

Une extension Moodle est programmée avec le langage PHP.
L'extension développée devrait pouvoir fonctionner avec plusieurs langues afin de pouvoir la déployer sur le répertoire d'extension Moodle.

Les mots-clés peuvent avoir des différences d'accords ou de conjugaison dans le texte de l'étudiant lorsqu'on les compare avec les mots-clés donnés par l'enseignant.
Il faut donc ramener les mots à leur forme la plus simple.
Deux techniques existent, la lemmatisation et la racination (stemming en anglais).

\begin{description}
  \item[Lemmatisation]
  
  La lemmatisation remet le mot à sa forme la plus simple (singulier, masculin, infinitif, etc.).
  Par exemple le verbe \og aimerait \fg{} sera remis à \og aimer \fg{}.
  
  \item[Racination]
  
  La racination enlève la fin du mot afin d'en conserver seulement la racine.
  Par exemple le verbe \og aimerait \fg{} sera remis à \og aim \fg{}.
\end{description}

La lemmatisation est une solution plus exacte que la racination, mais beaucoup plus complexe.
Mes recherches n'ont sorti aucune librairie PHP de lemmatisation fonctionnant avec plusieurs langues.
Par contre il existe une librairie de racination libre de droits appelés php-stemmer \cite{phpstemmer}.
Elle est sous \href{https://raw.githubusercontent.com/wamania/php-stemmer/master/LICENSE}{licence MIT} et utilise un algorithme développé par Dr Martin Porter écrite dans un langage appelé Snowball \cite{snowball}.
php-stemmer permet de faire la racination des mots en français, anglais, espagnol, allemand, italien, russe et plusieurs autres.
Le détail de l'algorithme Snowball est détaillé dans la section suivante.

Pour trouver la racine de tous les mots du texte, tous les caractères non alphanumériques sont remplacés par des espaces et le texte est découpé par les caractères d'espacements (espace, saut de ligne, tabulation, etc.).

\begin{lstlisting}[frame=l]
$words = preg_split('/(\s|\')/', preg_replace('/[^[:alnum:][:space:]]/u', ' ', $sentence));
\end{lstlisting}

Ensuite, chaque mot est associé avec sa racine trouvée avec l'algorithme Snowball.
Chaque mot-clé a, préalablement, aussi été réduit à leur racine avec l'algorithme Snowball.

\begin{lstlisting}[frame=l]
foreach ($words as $word) {
	if ($word) {
		if (Wamania\Snowball\Utf8::check($word)) {
			$stem = $stemmer->stem($word);
			if (isset($stems[$stem])) {
				if (!in_array($word, $stems[$stem])) {
					$stems[$stem][] = $word;
				}
			} else {
				$stems[$stem] = array($word);
			}
		} else {
			$stems[] = $word;
		}
	}
}
\end{lstlisting}

Finalement les mots-clés trouvés dans le texte sont mis en évidences.

\begin{lstlisting}[frame=l]
$usedKeywords = array_intersect(array_keys($stems), $keywords);

foreach ($usedKeywords as $keyword) {
	$words = $answerWords[$keyword];
	foreach ($words as $word) {
		$studentAnswer = str_replace($word, '<b><u>' . $word . '</u></b>', $studentAnswer);
	}
}
\end{lstlisting}

\section{Snowball}

Snowball est un petit langage de traitement de texte pour la racination de texte.
Une application Snowball peut être compilée en C ou en Java.
Le site web de Snowball donne l'algorithme de racination pour plusieurs langues: anglais, français, espagnol, allemand, russe, etc.
L'algorithme pour la langue anglaise se base sur l'algorithme de Porter.
L'auteur a créé une deuxième version en se basant sur les algorithmes des langues latines.
L'origine des algorithmes donnés pour les langues latines, ce qui inclut la langue française, n'est pas donnée, mais semble venir aussi de Porter avec l'aide de contributeurs.
La libraire PHP utilisé a été écrite en PHP par Wamania en se basant sur les algorithmes écrits en Snowball.
Le reste de cette section tentera d'expliquer la logique derrière l'algorithme de Snowball pour le français.
L'algorithme est séparé en 6 étapes en plus d'une étape de préparation et d'une étape finale.

\subsection*{Préparation}

Premièrement, le mot est transformé en minuscule par php-stemmer.
Les lettres u et i précédés et suivis par une voyelle, les lettres y précédé ou suivis par une voyelle ainsi que les u précédés de q sont mis en majuscules.
Les lettres en majuscules sont considérées en tant que consonnes dans l'algorithme.
Par exemple, le mot \og joUer \fg{}, le U ne sera pas considéré comme une minuscule.
Pour déterminer si une lettre est une voyelle, l'algorithme utilise une liste de lettre (a, e, i, o, u, y, â, à, ë, é, ê, è, ï, î, ô, û et ù).
Donc une voyelle en majuscule ne sortira pas de la liste.

Ensuite il faut trouver 3 sections dans le mot: RV (right vowel), R1 (right 1) et R2 (right 2).
Dans les exemples qui suivent, les lettres soulignées présenteront les lettres faisant partie du groupe.

\begin{description}
  \item[RV]
  
  Si le mot débute par deux voyelles, RV débute après la troisième lettre (oub\underline{lier}).
  Si le mot commence par \og par, col ou tap \fg{}, RV définit les lettres qui suivent (col\underline{onne}).
  Dans les autres cas RV débute après la première voyelle excluant la première lettre (bo\underline{njour}, algo\underline{rithme}).
  Si aucune de ses règles ne s'applique, RV est vide (un sigle comme IBN).
  RV n'existe pas dans l'algorithme de la langue anglaise.
  
  \item[R1]
  
  R1 débute après la première consonne précédée d'une voyelle (aim\underline{er}, tap\underline{is}).
  R1 est vide s'il n'y a pas de telle consonne.
  
  \item[R2]
  
  R2 débute après la première consonne précédée d'une voyelle dans R1 (fameus\underline{ement}).
  R2 est vide s'il n'y a pas de telle consonne.
\end{description}

RV peut contenir R1 et inversement.
Les lettres qui ne sont pas dans R1 ni dans RV ne seront pas touchés sauf pour quelques exceptions.

\subsection*{Étape 1: Suffixes standards}

Cette étape sert à trouver la racine des mots, adjectifs, adverbes, etc.

Trouver le suffixe le plus long parmi une liste donnée et faire l'action correspondante.
Voici des exemples de la liste qui compte 43 suffixes et 24 règles.

\begin{itemize}
  \item \textbf{ance iqUe isme able ...}: Supprime si dans R2
  \item \textbf{atrice ateur ation ...}: Supprime si dans R2. Si précédé de \textbf{ic}, supprimer si dans R2, sinon remplacer par \textbf{iqU}
  \item \textbf{logie logies}: Remplace par \textbf{log} si dans R2
  \item \textbf{eaux}: Remplace par \textbf{eau}
  \item \textbf{aux}: Remplace par \textbf{al} si dans R1
  \item \textbf{amment}: Remplacer par \textbf{ant} si dans RV
  \item \textbf{emment}: Remplacer par \textbf{ent} si dans RV
  \item \textbf{ment ments}: Supprimer si dans RV
\end{itemize}

\subsection*{Étape 2: Suffixes de verbes}

Cette étape trouve la racine d'un verbe.
On cherche et enlève les suffixes standards.

Faire cette étape seulement si l'étape 1 n'a rien changé ou si un des suffixes suivants a été trouvé: amment, emment, ment et ments.

\subsubsection*{Étape 2a: Suffixes débutant par i}

Trouver le suffixe le plus long parmi la liste ci-dessous.
S'il est précédé d'une consonne et que le tout est dans RV, supprimer le suffixe excluant la consonne supplémentaire.

îmes, ît, îtes, i, ie, ies, ir, ira, irai, iraIent etc.

\subsubsection*{Étape 2b: Autres suffixes de verbes}

Trouver le suffixe le plus long parmi les trois listes ci-dessous.
Les listes sont non-exaustives afin d'alléger le texte.

\begin{description}
  \item[ions]
  
  Supprimer si dans RV et R2
  
  \item[é ée ées és èrent er era erai eraIent ...]
  
  Supprimer si dans RV
  
  \item[âmes ât âtes a ai aIent ...]
  
  Supprimer si dans RV. Si la lettre e précède le suffixe et qu'elle se trouve aussi dans RV, la supprimer.
\end{description}

\subsection*{Étape 3: Dernière lettre}

Si les étapes 1 ou 2 ont modifié le mot, faire cette étape, sinon passer à l'étape 4.

Si la dernière lettre est un Y, la remplacer par i.
Si c'est un ç, la remplacer par c.

\subsection*{Étape 4: Résidue du suffixe}

Faire cette étape seulement si les étapes 1 à 3 n'ont pas modifié le mot.
Cette étape sert à enlever le pluriel et le féminin des mots très proches de leurs racines.

Si un mot se termine par \textbf{s} et n'est pas précédé de \textbf{a, i, o, u, è ou s}, supprimer ce s final.

Si un mot se termine par \textbf{ion}, que cette finale se trouve dans RV et dans R2 et qu'elle est précédée de la lettre \textbf{s ou t} (cette dernière doit se trouver dans RV), supprimer la finale \textbf{ion}.

Si un mot se termine par \textbf{ier, ière, Ier ou Ière} et que cette finale se trouve dans RV, remplacer cette dernière par \textbf{i}.

Si un mot se termine par \textbf{e} et que cette lettre ce trouve dans RV, la supprimer.

Si un mot se termine par \textbf{ë}, que cette lettre se trouve dans RV et qu'elle est précédée de \textbf{qu}, la supprimer.

\subsection*{Étape 5: Dédoubler la lettre finale}

Les suppressions et remplacements des étapes précédentes peuvent laisser une faute dans la racine du mot.
Cette étape sert à enlever les lettres doubles de certaines finales de mots.

Si un mot se termine par \textbf{enn, onn, ett, ell ou eill}, supprimer la dernière lettre.

\subsection*{Étape 6: Accent final}

Cette étape aussi sert a nettoyer la racine du mot.

Si un mot se termine par la lettre \textbf{è ou é} suivit d'une ou de plusieurs consonnes, enlever l'accent de ce \textbf{e}.

\subsection*{Finalisation}

Finalement, on enlève les majuscules sur les voyelles ajoutées durant l'étape de préparation.

Au final, nous avons la racine du mot.
Lors de la comparaison de texte, il peut y avoir quelques problèmes.
Par exemple les mots acceptables et accepté ont la même racine: accept.
Par contre le sens des 2 mots est différent.

\section{Tests}

Moodle offre deux types de tests: des tests d'acceptations et des tests unitaires.

\subsection{Tests d'acceptations}

Les tests d'acceptations sont exécutés à l'aide de Behat, un \og framwork \fg{} PHP d'automatisation de tests qui se base sur le \og Behavior Driven Developement (BDD) \fg{}.
Les tests sont écrits en anglais, compréhensibles par tous.
Chaque instruction et vérification doit être, préalablement, configurée en PHP.

Un serveur ou une application Selenium exécutera les tests directement dans l'interface d'un navigateur Web.

\begin{lstlisting}[frame=l]
@qtype @qtype_essay
Feature: Test creating an Essay question
  As a teacher
  In order to test my students
  I need to be able to create an Essay question

  Background:
    Given the following "users" exist:
      | username | firstname | lastname | email               |
      | teacher1 | T1        | Teacher1 | teacher1@moodle.com |
    And the following "courses" exist:
      | fullname | shortname | category |
      | Course 1 | C1        | 0        |
    And the following "course enrolments" exist:
      | user     | course | role           |
      | teacher1 | C1     | editingteacher |
    And I log in as "teacher1"
    And I follow "Course 1"
    And I navigate to "Question bank" node in "Course administration"

  Scenario: Create an Essay question with Response format set to 'HTML editor'
    When I add a "Essay" question filling the form with:
      | Question name            | essay-001                      |
      | Question text            | Write an essay with 500 words. |
      | General feedback         | This is general feedback       |
      | Response format          | HTML editor                    |
    Then I should see "essay-001"
\end{lstlisting}

\subsection{Tests unitaires}

Les tests unitaires sont exécutés avec PHP-Unit.


% \include{nom_de_fichier_du_chapitre_suivant}

\begin{conclusion}

Dans ce projet d'apparence plut\^ot simple, plusieurs points ont été sous-estim\'es.
Premi\`erement, cr\'eer un module d'extension Moodle ne consiste pas simplement \`a \'ecrire du code dans une fonction pr\'ed\'efinis ou \`a impl\'ementer des \textit{hook} comme plusieurs syst\`emes PHP (Wordpress et Typo3 par exemple).
Moodle est un syst\`eme complexe qui se r\'eflete dans son code et ses modules d'extensions.

Deuxi\`emement, l'organisation du temps \`a \'et\'e un grave probl\`eme.
Contrairement \`a des cours o\`u le projet doit \^etre r\'ealis\'e en quelques jours, ce projet s'\'etend sur plusieurs mois.
Le balancement de la vie priv\'ee, du travail et de la r\'ealisation de ce projet \`a \'et\'e un grave probl\`eme, au d\'epend du projet.

Derni\`erement, l'\'ecriture de ce rapport \`a \'et\'e beaucoup plus de travail que pr\'evu.
L'\'ecriture n'\'etant pas ma force et ayant rarement \'ecrit des textes de plus de 5 pages, l'\'ecriture de ce rapport \`a \'et\'e un beau d\'efi.

Ces difficult\'es vont m'aider dans ma vie professionnelle en:
\begin{itemize}
  \item Analysant mieux les syst\`emes inconnus avant de me lancer dans un projet;
  \item G\'erant mon horaire de fa\c{c}on plus \'efficace;
  \item M'aidant dans l'\'ecriture de rapports et de notes de cours.
\end{itemize}

Enseignant les cours de Web au c\'egep et en utilisant r\'eguli\`erement Moodle, les apprentissages amen\'es par ce projet vont m'aider dans ma carri\`ere.

\end{conclusion}

\GT{Selon le guide, tu es aussi suppos\'e traiter de <<R\'eflexion et
discussion de l'exp\'erience, en particulier sur l'atteinte des
objectifs et le transfert possible des connaissances et comp\'etences
acquises par l'\'etudiant dans le milieu professionnel;>>. Je crois
que la conclusion est un bon endroit pour le faire.}

% Utilisez l'environnement  conclusion pour rédiger votre conclusion

%%%%%%%%%%%%%%%%%%%%
% Page liminaires
%%%%%%%%%%%%%%%%%%%%

% \appendix
\chapter{Test d'acceptation existant dans Moodle}
\label{annexe_behat_exist}

\begin{lstlisting}[language=behat,frame=l]
@qtype @qtype_essay
Feature: Test creating an Essay question
  As a teacher
  In order to test my students
  I need to be able to create an Essay question

  Background:
    Given the following "users" exist:
      | username | firstname | lastname | email               |
      | teacher1 | T1        | Teacher1 | teacher1@moodle.com |
    And the following "courses" exist:
      | fullname | shortname | category |
      | Course 1 | C1        | 0        |
    And the following "course enrolments" exist:
      | user     | course | role           |
      | teacher1 | C1     | editingteacher |
    And I log in as "teacher1"
    And I am on "Course 1" course homepage
    And I navigate to "Question bank" node in "Course administration"

  Scenario: Create an Essay question with Response format set to 'HTML editor'
    When I add a "Essay" question filling the form with:
      | Question name            | essay-001                      |
      | Question text            | Write an essay with 500 words. |
      | General feedback         | This is general feedback       |
      | Response format          | HTML editor                    |
    Then I should see "essay-001"

  Scenario: Create an Essay question with Response format set to 'HTML editor with the file picker'
    When I add a "Essay" question filling the form with:
      | Question name            | essay-002                      |
      | Question text            | Write an essay with 500 words. |
      | General feedback         | This is general feedback       |
      | Response format          | HTML editor                    |
    Then I should see "essay-002"
\end{lstlisting}

\chapter{Test d'acceptation de \texttt{qtype\_essayhelper}}
\label{annexe_behat_new}

\begin{lstlisting}[language=behat,frame=l,style=default]
@qtype @qtype_essayhelper
Feature: Validate Essay with correction helper special features
  As a teacher
  In order to be helped correcting essays
  I need to see teacher answer and keywords while correcting while students don't

  Background:
    Given the following "users" exist:
      | username | firstname | lastname | email               |
      | teacher1 | T1        | Teacher1 | teacher1@moodle.com |
      | student1 | S1        | Student1 | student1@moodle.com |
    And the following "courses" exist:
      | fullname | shortname | category |
      | Course 1 | C1        | 0        |
    And the following "course enrolments" exist:
      | user     | course | role           |
      | teacher1 | C1     | editingteacher |
      | student1 | C1     | student        |
    And the following "question categories" exist:
      | contextlevel | reference | name           |
      | Course       | C1        | Test questions |
    And the following "questions" exist:
      | questioncategory | qtype       | name      | template |
      | Test questions   | essayhelper | essay-001 | plain    |
    And the following "activities" exist:
      | activity   | name   | course | idnumber |
      | quiz       | Quiz 1 | C1     | quiz1    |
    And quiz "Quiz 1" contains the following questions:
      | question   | page |
      | essay-001  | 1    |

  @javascript
  Scenario: A student submit an Essay with correction helper answer and don't see the correction helper.
    When I log in as "student1"
    And I am on "Course 1" course homepage
    And I follow "Quiz 1"
    And I press "Attempt quiz now"
    Then I should not see "Teacher answer" on quiz page "1"
    And I follow "Finish attempt ..."
    And I press "Submit all and finish"
    And I click on "Submit all and finish" "button" in the "Confirmation" "dialogue"
    Then I should see "Finished"
    Then I should not see "Teacher answer" on quiz page "1"

  @javascript
  Scenario: A teacher should see the correction helper while manually grading.
    # Create answer
    Given I log in as "student1"
    And I am on "Course 1" course homepage
    And I follow "Quiz 1"
    And I press "Attempt quiz now"
    And I set the field with xpath "//textarea[contains(@class, 'qtype_essayhelper_response')]" to "I think it's a frog"
    And I follow "Finish attempt ..."
    And I press "Submit all and finish"
    And I click on "Submit all and finish" "button" in the "Confirmation" "dialogue"
    And I log out

    # Go to manuel correction module
    And I log in as "teacher1"
    And I am on "Course 1" course homepage
    And I follow "Quiz 1"
    And I navigate to "Results > Manual grading" in current page administration
    And I should see "Manual grading"
    And I click on "grade all" "link" in the "essay-001" "table_row"
    Then I should see "Teacher answer"

    # Test Keyword
    Then "//b/u[contains(text(), 'frog')]" "xpath_element" should exist
\end{lstlisting}

\chapter{Exemple de test unitaire de \texttt{qtype\_essayhelper}}
\label{annexe_unittest}

\begin{lstlisting}[language=php,frame=l,style=default]
class qtype_essayhelper_stemmer_test extends basic_testcase {
    public function test_get_stemmer_all_languages_exists() {
        $stemmer = new qtype_essayhelper_stemmer();

        // Set get_stemmer function accessible
        $get_stemmer = $this->get_protected_function($stemmer, "get_stemmer");

        // Get all available languages
        $languages = PHPUnit\Framework\Assert::readAttribute($stemmer, "languages");

        // Test all languages
        foreach ($languages as $langCode => $lang) {
            $langStemmer = $get_stemmer->invokeArgs($stemmer, array($langCode));
            $this->assertEquals((new \ReflectionClass($langStemmer))->getShortName(), $lang);
        }
    }

    public function test_get_stemmer_language_non_existing() {
        $stemmer = new qtype_essayhelper_stemmer();

        // Set get_stemmer function accessible
        $get_stemmer = $this->get_protected_function($stemmer, "get_stemmer");

        // Test non existing language
        $langStemmer = $get_stemmer->invokeArgs($stemmer, array("zzz"));
        $this->assertEquals((new \ReflectionClass($langStemmer))->getShortName(), "English");
    }

    public function test_split_words_no_words() {
        $stemmer = new qtype_essayhelper_stemmer();
        $split_words = $this->get_protected_function($stemmer, "split_words");

        $this->assertEquals($split_words->invokeArgs($stemmer, array("")), array());
        $this->assertEquals($split_words->invokeArgs($stemmer, array("-\n    ' %")), array());
    }

    public function test_split_words_couple_words() {
        $stemmer = new qtype_essayhelper_stemmer();
        $split_words = $this->get_protected_function($stemmer, "split_words");

        $this->assertEquals($split_words->invokeArgs($stemmer, array("I love potatoes")), array("I", "love", "potatoes"));
        $this->assertEquals($split_words->invokeArgs($stemmer, array("I+love+potatoes")), array("I", "love", "potatoes"));
        $this->assertEquals($split_words->invokeArgs($stemmer, array("I-love\npotatoes")), array("I", "love", "potatoes"));
    }

    public function test_split_words_a_lot_of_words() {
        $stemmer = new qtype_essayhelper_stemmer();
        $split_words = $this->get_protected_function($stemmer, "split_words");

        $this->assertEquals($split_words->invokeArgs($stemmer,
            array("strap complex obtainable marked credit women wary educate nation wonder
              lours singulier musicien banniere lotus actrice premier polluer dans vie")),
            array("strap", "complex", "obtainable", "marked", "credit", "women", "wary", "educate",
                "nation", "wonder", "lours", "singulier", "musicien", "banni\`ere", "lotus", "actrice",
                "premier", "polluer", "dans", "vie"));
    }

    public function test_make_stem_array() {
        $stemmer = new qtype_essayhelper_stemmer();
        $make_stem_array = $this->get_protected_function($stemmer, "make_stem_array");

        // The stem for all words will be "test"
        $snowball_stemmer = $this->createMock('\\Wamania\\Snowball\\Stemmer', array('stem'));
        $snowball_stemmer->method('stem')->willReturn('test');

        $snowball_utf8 = new qtype_essayhelper_stemmer_utf8_mock();

        $make_stem_array = $this->get_protected_function($stemmer, 'make_stem_array');

        $stemmed_array = $make_stem_array->invokeArgs($stemmer, array("test asdf", $snowball_stemmer, $snowball_utf8));
        $expected_stemmed_array = array("test" => array("test", "asdf"));

        $this->assertEquals($stemmed_array, $expected_stemmed_array);
    }

    protected function get_protected_function($stemmer, $protectedFunctionName) {
        $reflection = new ReflectionClass($stemmer);
        $protectedFunction = $reflection->getMethod($protectedFunctionName);
        $protectedFunction->setAccessible(true);
        return $protectedFunction;
    }
}

class qtype_essayhelper_stemmer_utf8_mock extends Wamania\Snowball\Utf8 {
    public static function check($str) {
        return true;
    }
}
\end{lstlisting}

\nocite{*}
\bibliographystyle{apalike-uqam}
\bibliography{bibliographie}
\end{document}
