\chapter{Tests pour int\'egration \`a Moodle d'un module d'extension}
Moodle offre du support pour deux types de tests: des tests unitaires et des tests d'acceptation.
La plateforme inclut plusieurs centaines de ces deux types de tests automatis\'es d\`es son installation.
Ils ont tous \'et\'e effectu\'es avant le d\'ebut du d\'eveloppement afin de s'assurer que Moodle fonctionne correctement sans la pr\'esence du nouveau module d'extension.
Une fois le d\'eveloppement termin\'e, les tests ont \'et\'e effectu\'es de nouveau afin de valider que le nouveau module d'extension ne r\'egresse pas les autres fonctionnalit\'es de Moodle.

L'environnement utilis\'e afin de rouler les tests est une machine virtuelle Xubuntu~16.04 mise \`a jour en date du 14 d\'ecembre~2017.
La version de PHP est la version~7.0.22.
Le moteur de base de donn\'ees est MySQL version~5.7.20 install\'e avec les paquets Ubuntu.
La base de donn\'ees utilise la collation \texttt{utf8mb4\_unicode\_ci} telle que conseill\'ee dans la documentation\footnote{\url{https://docs.moodle.org/34/en/MySQL}}.
La version de Moodle est la derni\`ere version stable \`a ce jour, soit la version~3.4.
Moodle a \'et\'e install\'e avec \texttt{git} \`a partir de la branche \texttt{MOODLE\_34\_STABLE}, \textit{commit} a45c466\footnote{\url{https://github.com/moodle/moodle/commit/a45c46600021667691dbb4bce5420a2f65d3239c}} d\'eploy\'e le 14 d\'ecembre~2017.
Les tests ont \'et\'e ex\'ecut\'es une seule fois \textbf{avant} le d\'eveloppement du module d'extension afin de confirmer le fonctionnement de l'environnement.

\section{Tests unitaires}
\label{test-unitaires}
Les tests unitaires effectuent des v\'erifications \`a petite \'echelle.
Chaque fonction dans le code est test\'ee de fa\c{c}on isol\'ee.
Dans certains cas, pour tester de fa\c{c}on isol\'ee une fonction, il faut remplacer les d\'ependances par des \textit{stubs}.
Un \textit{stub} va simuler la d\'ependance utilis\'ee par la fonction en retournant une valeur fixe.
De cette mani\`ere, il est possible de tester cette fonction uniquement sans \^etre impact\'e par les autres fonctions.
Plusieurs tests peuvent \^etre effectu\'es afin de valider tous les cas possibles~\cite{tremblay16}.

\begin{lstfloat}[htbp]
\begin{lstlisting}[frame=l]
class qtype_essay_question_test extends advanced_testcase {
    public function test_get_question_summary() {
        $essay = test_question_maker::make_an_essay_question();
        $essay->questiontext = 'Hello <img src="http://example.com/globe.png" alt="world" />';
        $this->assertEquals('Hello [world]', $essay->get_question_summary());
    }
}
\end{lstlisting}
\caption{Exemple de test unitaire du module d'extension \texttt{qtype\_essay}.}
\label{code:unittest}
\end{lstfloat}

Une installation \og vanille \fg\ de Moodle vient avec plusieurs tests unitaires.
Le code de base ainsi que les modules d'extension de base sont d\'ej\`a test\'es.
Ces tests fonctionnent avec \textit{PHPUnit}, un \textit{framework} de tests unitaires pour PHP.
Un exemple de test unitaire est pr\'esent\'e dans l'extrait de code \ref{code:unittest}.
Cet exemple v\'erifie la validit\'e du r\'esum\'e de la question qui est le texte de la question sans images et sans mise en forme.
Le r\'esum\'e de la question va \^etre utilis\'e dans la banque de question, afin de la retrouver facilement dans une liste.


Avant de d\'eployer le module d'extension, la suite de tests \textit{PHPUnit}, fournie avec Moodle, a \'et\'e ex\'ecut\'ee.
Elle comporte 8750 tests et 90~575 v\'erifications (\textit{assertions}) \`a ex\'ecuter.
8682 tests ont r\'eussi (\textit{success}), 67 tests ont \'et\'e ignor\'es (\textit{skipped}) et un test a \'echou\'e (\textit{failures}).
Les tests ignor\'es sont, par exemple, les tests pour LDAP et les tests pour Redis, deux technologies qui n'ont pas \'et\'e configur\'ees dans notre cas.

Il y avait malheureusement une erreur caus\'ee par l'encodage des caract\`eres dans la base de donn\'ees MySQL.
Le test v\'erifie que la base de donn\'ees fonctionne en UTF8 et consid\`ere la casse (\textit{case-sensitive}) lors de la comparaison de texte.
La version de la base de donn\'ees MySQL utilis\'ee pour les tests est 5.7.
Or, il n'y a pas d'encodage UTF8 sensible \`a la casse pour les versions pr\'ec\'edentes \`a MySQL~5.8.
L'erreur est connue et d\'etaill\'ee sur le site de r\'ef\'erence Moodle\footnote{\url{https://docs.moodle.org/dev/Database_collation_issue}}.
Cette erreur peut poser probl\`eme dans les questions de type num\'erique ainsi que pour les remises de fichiers.
Par exemple, on ne peut pas entrer les unit\'es \og km \fg{} et \og Km \fg{} dans les r\'eponses possibles, car la base de donn\'ees consid\`ere ces unit\'es comme \'etant identiques.
Par contre, la correction de la question, ex\'ecut\'ee en PHP, distingue les deux unit\'es.
Un \'etudiant qui r\'epond avec l'unit\'e \og 10~KM \fg{} alors que l'enseignant a enregistr\'e la r\'eponse \og 10~km \fg{} aura donc une erreur.
Comme le module d'extension d\'evelopp\'e n'utilisera pas la comparaison de texte \`a partir de la base de donn\'ees, le d\'eveloppement a pu continuer sans probl\`eme.

Lorsque notre application est enti\`erement test\'ee avec des tests unitaires, nous sommes assur\'es qu'il ne devrait pas y avoir de probl\`eme de code dans les fonctions.
Par contre, si on compare les tests \`a une \'equipe sportive, les tests unitaires analysent chaque joueur, mais ne consid\`erent pas le travail d'\'equipe.
Il faut donc ajouter un autre type de test, les tests d'acceptation, afin de s'assurer de l'efficacit\'e du travail d'\'equipe.

\section{Tests d'acceptation}
\label{test_acceptation}
Un test d'acceptation valide que l'application satisfait aux exigences du logiciel.
Ce type de test peut s'ex\'ecuter \`a partir de l'interface, de la m\^eme mani\`ere qu'un humain pourrait tester le logiciel --- il s'agit donc d'un test qui ex\'ecute le logiciel <<de bout en bout>>~\cite{tremblay16}.

Moodle vient aussi avec plusieurs de ces tests, mais ce ne sont pas tous les modules d'extension qui en ont.
%
Les tests d'acceptation sont ex\'ecut\'es \`a l'aide de \texttt{behat}, un \textit{framework} PHP d'automatisation de tests qui se base sur le \og \textit{Behavior Driven Development} (BDD) \fg{}~\cite{tremblay16}.



Un serveur ou une application Selenium ex\'ecutera les tests en ligne de commande ou dans un navigateur web, selon le besoin.
Les tests sont \'ecrits sous forme textuelle anglaise, g\'en\'eralement compr\'ehensibles par tous les intervenants.
Chaque instruction et v\'erification doit \^etre, pr\'ealablement, configur\'ee en PHP.
La structure des tests d'acceptation avec \texttt{behat} est comme suit (un exemple se trouve à l'annexe~\ref{annexe_behat_exist}, p.~\pageref{annexe_behat_exist}):
\begin{itemize}
  \item La premi\`ere ligne avec un \verb|@| permet de cat\'egoriser chaque fichier de test.
        Par exemple, ceci permet d'ex\'ecuter les tests d'acceptations pour un seul module d'extension ou pour un type de module d'extension;
        
  \item La ligne d\'ebutant par \textit{Feature} donne le titre (le nom) du test afin de le retrouver facilement;
  
  \item Les trois lignes suivantes permettent de d\'ecrire le test au lecteur, \`a un haut niveau d'abstraction.
        Comme dans de nombreuses autres formes de tests d'acceptation, on les \'ecrit habituellement comme suit, donc \`a l'aide d'un r\'ecit utilisateur (\emph{user story}):
        
        \begin{itemize}
          \item \og \textit{As a ...} \fg{} d\'ecrit quel type d'utilisateur est cibl\'e par ce test;
          \item \og \textit{In order to ...} \fg{} d\'ecrit l'action \`a tester;
          \item \og \textit{I need to ...} \fg{} d\'ecrit ce qu'il faut v\'erifier.
        \end{itemize}
        
  \item Ensuite, on d\'ebute le \textit{Background} qui pr\'epare le sc\'enario sp\'ecifique de test.
        La premi\`ere action de cette section d\'ebutera par \textit{Given} et toutes les autres par \textit{And}.
        Chaque action pr\'epare l'environnement pour le sc\'enario de test, par exemple, ajouter un enregistrement dans la base de donn\'ees, naviguer \`a une certaine page, cliquer sur un bouton, etc;
        
  \item Ensuite, il y a une ou plusieurs sections \textit{Scenario} qui d\'efinissent chaque sc\'enario sp\'ecifique de test \`a effectuer.
        Sur la m\^eme ligne que le \textit{Scenario}, il y a une description du test pour le lecteur.
        Ensuite, le \textit{When} d\'efinit l'action \`a ex\'ecuter pour ce sc\'enario de test.
        Finalement, le \textit{Then} d\'efinit le comportement attendu suite \`a l'ex\'ecution de l'action pour ce sc\'enario.
\end{itemize}

Une particularité de \texttt{behat} est qu'il permet d'avoir un seul \textit{Given} (\textit{Background}) pour plusieurs \textit{When} et \textit{Then} (\textit{Scenario}), donc
le \textit{Background} est ex\'ecut\'e avant chaque \textit{Scenario}\footnote{\url{http://docs.behat.org/en/latest/user\_guide/writing\_scenarios.html\#user-guide-writing-scenarios-backgrounds}}.

La s\'erie de tests d'acceptation de Moodle a \'et\'e ex\'ecut\'ee avant le d\'eveloppement du module d'extension.
Il y a un total de 1~771 sc\'enarios pour un total de 43~824~\'etapes.
Parmi ceux-ci, 4 sc\'enarios et 102 \'etapes ont \'et\'e ignor\'es et 6 \'etapes et 6 sc\'enarios ont \'echou\'e.
Voici une description des erreurs rencontr\'ees ainsi que leur influence sur le d\'eveloppement dans ce projet:
\begin{enumerate}
  \item Erreur lorsqu'un \'etudiant passe d'une activit\'e \`a une autre.
        Notre module d'extension se concentre sur une seule activit\'e, ce cas n'est donc pas probl\'ematique.
        
  \item Erreur dans le filtre du calendrier mensuel.
        Notre module d'extension ne touche pas au calendrier, ce cas n'est donc pas probl\'ematique.
        
  \item Erreur dans la navigation entre les modes de groupes.
        Notre module d'extension ne touche pas aux modes de groupes, ce cas n'est donc pas probl\'ematique.
        
  \item \texttt{Solr} (engin de recherche) n'est pas install\'e sur l'environnement de test.
        Notre module d'extension ne touche pas \`a la recherche, ce cas n'est donc pas probl\'ematique.
        
  \item Erreur \texttt{Solr} identique \`a la pr\'ec\'edente.
  
  \item Erreur dans la liste des \'etudiants, l'enseignant ne voit pas quels \'etudiants sont actifs.
        Notre module d'extension ne touche pas \`a la liste des \'etudiants, ce cas n'est donc pas probl\'ematique.
\end{enumerate}
Comme aucun des six cas n'est probl\'ematique, le d\'eveloppement a pu se poursuivre sans probl\`eme.
Lors de l'ex\'ecution finale des tests, il ne devrait y avoir que ces six m\^emes erreurs.

\section{Tests dans un module d'extension}
Chaque module d'extension doit aussi d\'efinir ses propres tests unitaires et tests d'acceptation.
Notre nouveau module d'extension devra donc lui aussi d\'efinir des tests appropri\'es afin de valider son bon fonctionnement.
Ces tests pour notre module sont d\'ecrits \`a la section~\ref{dev_test}.
