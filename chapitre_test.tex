\chapter{Tests avec Moodle}
Moodle offre du support pour deux types de tests: des tests unitaires et des tests d'acceptation.
La plateforme inclut plusieurs centaines de ces deux types de tests automatis\'es d\`es son installation.
Ils seront tous effectu\'es avant le d\'ebut du d\'eveloppement afin de s'assurer que Moodle fonctionne correctement sans le nouveau module d'extension.
Une fois le d\'eveloppement termin\'e, les tests seront effectu\'es de nouveau afin de valider que le nouveau module d'extension ne r\'egresse pas les autres fonctionnalit\'es de Moodle.
L'environnement utilis\'e afin de rouler les tests est une machine virtuelle Xubuntu~16.04 mise \`a jour en date du 14 d\'ecembre 2017.
La version de PHP est 7.0.22.
Le moteur de base de donn\'ees est MySQL version~5.7.20 install\'e avec les paquets Ubuntu.
La base de donn\'ees utilise la collation \textit{utf8mb4\_unicode\_ci} comme conseill\'ee dans la documentation\footnote{\url{https://docs.moodle.org/34/en/MySQL}}.
La version de Moodle est la derni\`ere version stable \`a ce jour, soit la version~3.4.
Moodle a \'et\'e install\'e avec \texttt{git} \`a partir de la branche \textit{MOODLE\_34\_STABLE}, \textit{commit} a45c466\footnote{\url{https://github.com/moodle/moodle/commit/a45c46600021667691dbb4bce5420a2f65d3239c}} d\'eploy\'e le 14 d\'ecembre 2017.
Les tests ont \'et\'e ex\'ecut\'es une seule fois avant le d\'eveloppement du module d'extension afin de confirmer le fonctionnement de l'environnement.
\section{Tests unitaires}
Les tests unitaires effectuent des v\'erifications \`a petite \'echelle.
Chaque fonction dans le code est test\'ee de fa\c{c}on isol\'ee.
Dans certains cas, pour tester de fa\c{c}on isol\'ee une fonction, il faut remplacer les d\'ependances par des \textit{mocks}.
Un \textit{mock} va simuler la d\'ependance utilis\'ee par la fonction en retournant une valeur fixe.
De cette mani\`ere, il est possible de tester cette fonction uniquement sans \^etre impact\'e par les autres fonctions.
Plusieurs tests peuvent \^etre effectu\'es afin de valider tous les cas possibles. \cite{tremblay16}
Une installation \og vanille \fg\ de Moodle vient avec plusieurs tests unitaires.
Le code de base ainsi que les modules d'extensions de base sont d\'ej\`a test\'es.
Ces tests fonctionnent avec \textit{PHPUnit}, un \textit{framework} de tests unitaires pour PHP.
Un exemple de test unitaire est illustr\'e \`a l'extrait de code \ref{code:unittest}.
Cet exemple v\'erifie la validit\'e du r\'esum\'e de la question qui est le texte de la question sans images et sans mise en forme.
Le r\'esum\'e de la question va \^etre utilis\'e dans la banque de question, afin de la retrouver facilement dans une liste.
\begin{lstfloat}
\begin{lstlisting}[frame=l]
class qtype_essay_question_test extends advanced_testcase {
    public function test_get_question_summary() {
        $essay = test_question_maker::make_an_essay_question();
        $essay->questiontext = 'Hello <img src="http://example.com/globe.png" alt="world" />';
        $this->assertEquals('Hello [world]', $essay->get_question_summary());
    }
}
\end{lstlisting}
\caption{Exemple de test unitaire du module d'extension \textit{qtype\_essay}.}
\label{code:unittest}
\end{lstfloat}
Avant de d\'eployer le module d'extension, la suite de tests \textit{PHPUnit}, fournis de base avec Moodle, a \'et\'e ex\'ecut\'ee.
Il y a 8750 tests et 90~575 v\'erifications (\textit{assertions}) \`a ex\'ecuter.
8682 tests ont r\'eussi (\textit{success}), 67 tests ont \'et\'e ignor\'es (\textit{skipped}) et un test a \'echou\'e (\textit{failures}).
Les tests ignor\'es sont, par exemple, les tests pour LDAP et les tests pour Redis, deux technologies qui n'ont pas \'et\'e configur\'ees dans notre cas.
Il y avait malheureusement une erreur caus\'ee par l'encodage des caract\`eres dans la base de donn\'ees MySQL.
Le test v\'erifie que la base de donn\'ees fonctionne en UTF8 et consid\`ere la casse (\textit{case-sensitive}) lors de la comparaison de texte.
La version de la base de donn\'ees MySQL utilis\'ee pour les tests est 5.7.
Or, il n'y a pas d'encodage UTF8 sensible \`a la casse pour les versions pr\'ec\'edentes \`a MySQL~5.8.
L'erreur est connue et d\'etaill\'ee sur le site de r\'ef\'erence Moodle\footnote{\url{https://docs.moodle.org/dev/Database_collation_issue}}.
Cette erreur peut poser probl\`eme dans les questions de type num\'erique ainsi que pour les remises de fichiers.
Par exemple, on ne peut pas entrer les unit\'es \og km \fg{} et \og Km \fg{} dans les r\'eponses possibles, car la base de donn\'ees consid\`ere ces unit\'es comme \'etant identiques.
Par contre, la correction de la question, ex\'ecut\'ee en PHP, fait la diff\'erence entre les deux unit\'es.
Un \'etudiant qui r\'epond avec l'unit\'e \og 10~KM \fg{} alors que l'enseignant a enregistr\'e la r\'eponse \og 10~km \fg{} aura une erreur.
Comme le module d'extension d\'evelopp\'e n'utilisera pas la comparaison de texte \`a partir de la base de donn\'ees, le d\'eveloppement peut continuer sans probl\`eme.
Si notre application est enti\`erement test\'ee avec des tests unitaires, nous sommes assur\'es qu'il ne devrait pas y avoir de probl\`eme de code dans les fonctions.
Par contre, si on compare les tests \`a une \'equipe sportive, les tests unitaires analysent chaque joueur, mais ne consid\`erent pas le travail d'\'equipe.
Il faut donc ajouter un autre type de test, les tests d'acceptation, afin de s'assurer de l'efficacit\'e du travail d'\'equipe.
\section{Tests d'acceptation}
Un test d'acceptation valide que l'application satisfait aux exigences du logiciel.
Ce type de test peut s'ex\'ecuter \`a partir de l'interface, de la m\^eme mani\`ere qu'un humain pourrait tester le logiciel --- il s'agit donc d'un test qui ex\'ecute le logiciel <<de bout en bout>>.  \cite{tremblay16}

\GT{Il faudrait donner une r\'ef\'erence pour tests unitaires, BDD, etc.}
\PG{J'ai ajout\'e des r\'ef\'erences vers vos notes de cours. J'ai un peu ajust\'e mes d\'efinitions apr\`es mes lectures.}
Moodle vient aussi avec plusieurs de ces tests, mais ce ne sont pas tous les modules d'extensions qui en ont.
%
Les tests d'acceptation sont ex\'ecut\'es \`a l'aide de \texttt{behat}, un \og framework \fg{} PHP d'automatisation de tests qui se base sur le \og \textit{Behavior Driven Development} (BDD) \fg{}. \PG{R\'ef\'erence BDD \`a venir...}
Un serveur ou une application Selenium ex\'ecutera les tests en ligne de commande ou dans un navigateur web, selon le besoin.
Les tests sont \'ecrits en anglais, compr\'ehensibles par tous.
Chaque instruction et v\'erification doit \^etre, pr\'ealablement, configur\'ee en PHP.
La structure des tests d'acceptation avec \texttt{behat} est comme suit (un exemple se trouve \`a l'extrait de code \ref{code:behattest}):
\begin{itemize}
  \item La premi\`ere ligne avec les \verb|@| permet de cat\'egoriser chaque fichier de test.
        Ceci permet d'ex\'ecuter les tests d'acceptations pour un seul module d'extension ou pour un type de module d'extension;
        
  \item D\'ebutant par \textit{Feature}, le titre du test afin de le retrouver facilement;
  
  \item Les trois lignes suivantes permettent de d\'ecrire le test au lecteur.
        En Moodle, comme dans de nombreuses autres formes de tests d'acceptation, on les \'ecrit habituellement comme suit:
        
        \begin{itemize}
          \item \og \textit{As a ...} \fg{} d\'ecrit quel type d'utilisateur est cibl\'e par ce test;
          \item \og \textit{In order to ...} \fg{} d\'ecrit l'action \`a tester;
          \item \og \textit{I need to ...} \fg{} d\'ecrit ce qu'il faut v\'erifier.
        \end{itemize}
        
  \item Ensuite, on d\'ebute le \textit{Background} qui pr\'epare le test.
        La premi\`ere action de cette section d\'ebutera par \textit{Given} et toutes les autres par \textit{And}.
        Chaque action pr\'epare l'environnement pour le test, par exemple: ajouter un enregistrement dans la base de donn\'ees, naviguer \`a une certaine page, cliquer sur un bouton, etc;
        
  \item Ensuite, il y a une ou plusieurs sections \textit{Scenario} qui d\'efinissent chaque test \`a effectuer.
        Sur la m\^eme ligne que le \textit{Scenario}, il y a une description du test pour le lecteur.
        Ensuite, le \textit{When} d\'efinit l'action \`a tester.
        Finalement, le \textit{Then} d\'efinit le comportement entendu suite au test.
\end{itemize}
\begin{lstfloat}
\begin{lstlisting}[frame=l]
@qtype @qtype_essay
Feature: Test creating an Essay question
  As a teacher
  In order to test my students
  I need to be able to create an Essay question
  Background:
    Given the following "users" exist:
      | username | firstname | lastname | email               |
      | teacher1 | T1        | Teacher1 | teacher1@moodle.com |
    And the following "courses" exist:
      | fullname | shortname | category |
      | Course 1 | C1        | 0        |
    And the following "course enrolments" exist:
      | user     | course | role           |
      | teacher1 | C1     | editingteacher |
    And I log in as "teacher1"
    And I follow "Course 1"
    And I navigate to "Question bank" node in "Course administration"
  Scenario: Create an Essay question with Response format set to 'HTML editor'
    When I add a "Essay" question filling the form with:
      | Question name            | essay-001                      |
      | Question text            | Write an essay with 500 words. |
      | General feedback         | This is general feedback       |
      | Response format          | HTML editor                    |
    Then I should see "essay-001"
\end{lstlisting}
\caption{Test d'acceptation du module d'extension \textit{qtype\_essay}.}
\label{code:behattest}
\end{lstfloat}
La s\'erie de tests d'acceptation de Moodle a \'et\'e ex\'ecut\'ee avant le d\'eveloppement du module d'extension.
Il y a un total de 1~771 sc\'enarios pour un total de 43~824~\'etapes.
Parmi ceux-ci, 4 sc\'enarios et 102 \'etapes ont \'et\'e ignor\'es et 6 \'etapes et 6 sc\'enarios ont \'echou\'e.
Voici une description des erreurs rencontr\'ees ainsi que leur influence sur le d\'eveloppement dans ce projet:
\begin{enumerate}
  \item Erreur lorsqu'un \'etudiant passe d'une activit\'e \`a une autre.
        Notre module d'extension se concentre sur une seule activit\'e, ce cas n'est donc pas probl\'ematique.
        
  \item Erreur dans le filtre du calendrier mensuel.
        Notre module d'extension ne touche pas au calendrier, ce cas n'est donc pas probl\'ematique.
        
  \item Erreur dans la navigation entre les modes de groupes.
        Notre module d'extension ne touche pas aux modes de groupes, ce cas n'est donc pas probl\'ematique.
        
  \item \texttt{Solr} (engin de recherche) n'est pas install\'e sur l'environnement de test.
        Notre module d'extension ne touche pas \`a la recherche, ce cas n'est donc pas probl\'ematique.
        
  \item Erreur \texttt{Solr} identique \`a la pr\'ec\'edente.
  
  \item Erreur dans la liste des \'etudiants, l'enseignant ne voit pas quels \'etudiants sont actifs.
        Notre module d'extension ne touche pas \`a la liste des \'etudiants, ce cas n'est donc pas probl\'ematique.
\end{enumerate}
Comme aucun des six cas n'est probl\'ematique, le d\'eveloppement peut se poursuivre sans probl\`eme.
Lors de l'ex\'ecution finale des tests, il ne devrait y avoir que ces six m\^emes erreurs.
\GT{De ce que tu \'ecrit plus haut, ce n'est pas clair si tu as
uniquement ex\'ecut\'e les tests d\'ej\`a disponibles ou si tu en as
toi-m\^eme d\'efini d'autres.  De ce qu'on comprend de ce que tu
\'ecris dans le chapitre suivant, tu en as d\'efini donc il faudrait
clarifier cela d\`es \`a pr\'esent.}
\section{Tests dans un module d'extension}
Chaque module d'extension d\'efinit ces propres tests unitaires et d'acceptation.
Le nouveau module d'extension devra donc lui aussi d\'efinir des tests unitaires et d'acceptation afin de valider son bon fonctionnement.
L'ajout de tests est d\'etaill\'e au \`a la section \ref{dev_test}.