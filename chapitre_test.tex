\chapter{Tests avec Moodle}

Moodle offre deux types de tests: des tests d'acceptation et des tests unitaires.

\section{Tests d'acceptation}

TODO: Expliquer les tests d'acception et leur utilisation dans Moodle

Les tests d'acceptation sont exécutés à l'aide de Behat, un \og framwork \fg{} PHP d'automatisation de tests qui se base sur le \og \textit{Behavior Driven Development} (BDD) \fg{}.
Les tests sont écrits en anglais, compréhensibles par tous.
Chaque instruction et vérification doit être, préalablement, configurée en PHP.

Un serveur ou une application Selenium exécutera les tests directement dans l'interface d'un navigateur Web.

\begin{lstfloat}
\begin{lstlisting}[frame=l]
@qtype @qtype_essay
Feature: Test creating an Essay question
  As a teacher
  In order to test my students
  I need to be able to create an Essay question

  Background:
    Given the following "users" exist:
      | username | firstname | lastname | email               |
      | teacher1 | T1        | Teacher1 | teacher1@moodle.com |
    And the following "courses" exist:
      | fullname | shortname | category |
      | Course 1 | C1        | 0        |
    And the following "course enrolments" exist:
      | user     | course | role           |
      | teacher1 | C1     | editingteacher |
    And I log in as "teacher1"
    And I follow "Course 1"
    And I navigate to "Question bank" node in "Course administration"

  Scenario: Create an Essay question with Response format set to 'HTML editor'
    When I add a "Essay" question filling the form with:
      | Question name            | essay-001                      |
      | Question text            | Write an essay with 500 words. |
      | General feedback         | This is general feedback       |
      | Response format          | HTML editor                    |
    Then I should see "essay-001"
\end{lstlisting}
\caption{Test d'acceptation du module d\'extension \og qtype\_essay \fg{}.}
\label{code:commentaire}
\end{lstfloat}

\section{Tests unitaires}

TODO: Expliquer les tests unitaires et leur utilisation dans Moodle

Les tests unitaires sont exécutés avec PHP-Unit.