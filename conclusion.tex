\begin{conclusion}
Ce projet m'a permis de mettre en pratique quelques notions \'etudi\'ees durant mon parcours universitaire.
Une de ces notions est la r\'edaction de tests unitaires, pratique courante dans le milieu professionnel.
Ce type de test a seulement \'et\'e pratiqu\'e durant mon baccalaur\'eat au cours d'une p\'eriode o\`u on devait essayer le \texttt{TDD} (\textit{Test Driven Developement}).
Dans la m\^eme cat\'egorie, il y a les tests d'acceptation.
Ce type de test a seulement \'et\'e cit\'e dans mes cours, j'ai finalement vu ce type de test en action en plus d'en \'ecrire quelques-uns.
Finalement, ce document est r\'edig\'e en \texttt{LaTeX}, technologie d'ont je ne connaissais que le nom au d\'ebut du projet.
Avec l'aide de M. Tremblay, j'ai pu en apprendre beaucoup plus sur ce sujet durant la r\'edaction de ce rapport.

J'ai aussi beaucoup appris sur Moodle, plateforme utilis\'ee dans mon emploi au C\'egep R\'egional de Lanaudi\`ere \`a Joliette.
Ces nouvelles connaissances vont m'\^etre utilent afin de d\'evelopper des outils pour la communaut\'e coll\'egiale telle qu'une liste de travaux o\`u la progression de l'\'etudiant pourra \^etre compar\'e avec le reste du groupe.

Dans ce projet, initialement d'apparence plut\^ot simple, plusieurs points ont \'et\'e sous-estim\'es.
Premi\`erement, la cr\'eation de modules d'extension Moodle diff\`ere de mon exp\'erience en mati\`ere de module d'extension \texttt{PHP} avec Wordpress et Typo3.
Moodle est un syst\`eme complexe, ce qui se refl\`ete dans son code et ses modules d'extension.
L'interaction entre le module d'extension et Moodle n'est pas toujours claire et aurait d\^u \^etre analys\'ee plus en d\'etail au d\'ebut du d\'eveloppement.
Deuxi\`emement, l'organisation et la planification du temps a \'et\'e une source de difficult\'e.
Contrairement \`a un cours o\`u le projet peut/doit \^etre r\'ealis\'e en quelques jours --- au pire quelques semaines --- , ce projet s'est \'etendu sur plusieurs mois.
L'\'equilibre entre travail r\'emun\'er\'e,  vie priv\'ee et r\'ealisation de ce projet acad\'emiqe a \'et\'e une source de difficult\'e, parfois au d\'epend du projet.
Finalement, l'\'ecriture de ce rapport a n\'ecessit\'e plus de travail que pr\'evu.
L'\'ecriture n'\'etant pas ma force et ayant rarement \'ecrit des textes de plus de 5--10 pages, l'\'ecriture de ce rapport a \'et\'e un important d\'efi.
Ces difficult\'es que j'ai rencontr\'ees et en partie surmont\'ees vont, je l'esp\`ere, m'aider dans ma vie professionnelle:
\begin{itemize}
  \item \`A mieux analyser les syst\`emes inconnus avant de me lancer dans un projet;
  \item \`A mieux g\'erer mon horaire de fa\c{c}on plus efficace;
  \item \`A m'aider dans l'\'ecriture de rapports et de notes de cours.
\end{itemize}
J'esp\`ere aussi qu'en enseignant les cours de Web au c\'egep et en utilisant r\'eguli\`erement Moodle, les apprentissages faits dans le cadre de ce projet vont m'aider dans ma carri\`ere d'enseignant.
\end{conclusion}