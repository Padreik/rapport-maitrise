\begin{conclusion}

Dans ce projet d'apparence plut\^ot simple, plusieurs points ont été sous-estim\'es.
Premi\`erement, cr\'eer un module d'extension Moodle ne consiste pas simplement \`a \'ecrire du code dans une fonction pr\'ed\'efinis ou \`a impl\'ementer des \textit{hook} comme plusieurs syst\`emes PHP (Wordpress et Typo3 par exemple).
Moodle est un syst\`eme complexe qui se r\'eflete dans son code et ses modules d'extensions.

Deuxi\`emement, l'organisation du temps \`a \'et\'e un grave probl\`eme.
Contrairement \`a des cours o\`u le projet doit \^etre r\'ealis\'e en quelques jours, ce projet s'\'etend sur plusieurs mois.
Le balancement de la vie priv\'ee, du travail et de la r\'ealisation de ce projet \`a \'et\'e un grave probl\`eme, au d\'epend du projet.

Derni\`erement, l'\'ecriture de ce rapport \`a \'et\'e beaucoup plus de travail que pr\'evu.
L'\'ecriture n'\'etant pas ma force et ayant rarement \'ecrit des textes de plus de 5 pages, l'\'ecriture de ce rapport \`a \'et\'e un beau d\'efi.

Ces difficult\'es vont m'aider dans ma vie professionnelle en:
\begin{itemize}
  \item Analysant mieux les syst\`emes inconnus avant de me lancer dans un projet;
  \item G\'erant mon horaire de fa\c{c}on plus \'efficace;
  \item M'aidant dans l'\'ecriture de rapports et de notes de cours.
\end{itemize}

Enseignant les cours de Web au c\'egep et en utilisant r\'eguli\`erement Moodle, les apprentissages amen\'es par ce projet vont m'aider dans ma carri\`ere.

\end{conclusion}

\GT{Selon le guide, tu es aussi suppos\'e traiter de <<R\'eflexion et
discussion de l'exp\'erience, en particulier sur l'atteinte des
objectifs et le transfert possible des connaissances et comp\'etences
acquises par l'\'etudiant dans le milieu professionnel;>>. Je crois
que la conclusion est un bon endroit pour le faire.}
