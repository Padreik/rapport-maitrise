\begin{conclusion}

L'affichage de la réponse de l'enseignant et de l'étudiant côte à côte en plus des instructions au correcteur devraient aider le correcteur dans sa tâche de correction.

La mise en évidence des mots clés est encore simple, elle pourrait être configurable via l'interface d'administration afin de changer le style de mise en évidence.
La racination utilisée est aussi limitée~: le correcteur doit lire le contexte afin de confirmer si le mot-clé est valide, problème qui pourrait être éliminé avec un outil de lemmatisation.
Finalement, les mots-clés en soit ne sont pas affichés au correcteur, ce qui pourrait être un ajout intéressant.

La racination des mots n'est pas configurable non plus, elle doit être \og hardcodée \fg{} dans le code.
Il faudrait permettre la configuration de la langue dans la configuration de la question, permettant ainsi à un enseignant d'anglais d'utiliser des mots-clés en anglais même si la communauté est francophone.
La langue par défaut devrait être configurable par l'administrateur.

Dans la planification du projet, il était prévu de pouvoir comparer les textes des étudiants entre eux (au lieu de seulement comparer à l'enseignant).
Par contre, cette idée demandait une analyse syntaxique qui sortait du cadre du projet.
Cet ajout pourrait être un projet intéressant pour un futur étudiant.

Toutes ces limitations et correctifs à apporter mettent en doute la complétude du module d'extension développé.
Il ne sera donc pas soumis au répertoire des modules d'extensions Moodle.
Après améliorations, il pourrait toutefois devenir un ajout intéressant pour les enseignants et correcteurs du monde entier.

Le code source est disponible sur GitHub comme le stipule les règles de développement Moodle.
Il se trouve à l'adresse \url{https://github.com/Padreik/moodle-qtype_essayhelper}.

\end{conclusion}
