\chapter{Détection des mots clés}

TODO: Introduire la lemmatisation et la racination

\section{La lemmatisation}

TODO: Décrire la lemmatisation et pourquoi on ne l'utilise pas

\section{La racination}

TODO: Décrire la racination et pourquoi on l'utilise

\section{Snowball}

Snowball est un petit langage de traitement de texte pour la racination de texte.
Une application Snowball peut être compilée en C ou en Java.
Le site web de Snowball donne l'algorithme de racination pour plusieurs langues: anglais, français, espagnol, allemand, russe, etc.
L'algorithme pour la langue anglaise se base sur l'algorithme de Porter.
L'auteur a créé une deuxième version en se basant sur les algorithmes des langues latines.
L'origine des algorithmes donnés pour les langues latines, ce qui inclut la langue française, n'est pas donnée, mais semble venir aussi de Porter avec l'aide de contributeurs.
La libraire PHP utilisée a été écrite en PHP par Wamania en se basant sur les algorithmes écrits en Snowball.
Le reste de cette section tente d'expliquer la logique derrière l'algorithme de Snowball pour le français.
L'algorithme est séparé en six étapes en plus d'une étape de préparation et d'une étape finale.

\subsection*{Préparation}

Premièrement, le mot est transformé en minuscule par php-stemmer.
Les lettres u et i précédées et suivies par une voyelle, les lettres y précédées ou suivies par une voyelle ainsi que les u précédées de q sont mises en majuscules.
Les lettres en majuscules sont considérées en tant que consonnes dans l'algorithme.
Par exemple, dans le mot \og joUer \fg{}, le U ne sera pas considéré comme une minuscule.
Pour déterminer si une lettre est une voyelle, l'algorithme utilise une liste de lettres (a, e, i, o, u, y, â, à, ë, é, ê, è, ï, î, ô, û et ù).
Donc une voyelle en majuscule ne fera pas partie de la liste.

Ensuite il faut trouver trois sections dans le mot: RV (right vowel), R1 (right 1) et R2 (right 2).
Dans les exemples qui suivent, les lettres soulignées indiquent les lettres faisant partie du groupe.

\begin{description}
  \item[RV]
  
  Si le mot débute par deux voyelles, RV débute après la troisième lettre (oub\underline{lier}).
  Si le mot commence par \og par, col ou tap \fg{}, RV définit les lettres qui suivent (col\underline{onne}).
  Dans les autres cas, RV débute après la première voyelle excluant la première lettre (bo\underline{njour}, algo\underline{rithme}).
  Si aucune de ses règles ne s'applique, RV est vide (un sigle comme IBN).
  RV n'existe pas dans l'algorithme de la langue anglaise.
  
  \item[R1]
  
  R1 débute après la première consonne précédée d'une voyelle (aim\underline{er}, tap\underline{is}).
  R1 est vide s'il n'y a pas de telle consonne.
  
  \item[R2]
  
  R2 débute après la première consonne précédée d'une voyelle dans R1 (fameus\underline{ement}).
  R2 est vide s'il n'y a pas de telle consonne.
\end{description}

RV peut contenir R1 et inversement.
Les lettres qui ne sont pas dans R1 ni dans RV ne seront pas touchées sauf pour quelques exceptions.

\subsection*{Étape 1: Suffixes standards}

Cette étape sert à trouver la racine des mots, adjectifs, adverbes, etc.

Trouver le suffixe le plus long parmi une liste donnée et faire l'action correspondante.
Voici des exemples de la liste qui compte 43 suffixes et 24 règles.

\begin{itemize}
  \item \textbf{ance iqUe isme able ...}: Supprimer si dans R2;
  \item \textbf{atrice ateur ation ...}: Supprimer si dans R2. Si précédé de \textbf{ic}, supprimer si dans R2, sinon remplacer par \textbf{iqU};
  \item \textbf{logie logies}: Remplacer par \textbf{log} si dans R2;
  \item \textbf{eaux}: Remplacer par \textbf{eau};
  \item \textbf{aux}: Remplacer par \textbf{al} si dans R1;
  \item \textbf{amment}: Remplacer par \textbf{ant} si dans RV;
  \item \textbf{emment}: Remplacer par \textbf{ent} si dans RV;
  \item \textbf{ment ments}: Supprimer si dans RV.
\end{itemize}

\subsection*{Étape 2: Suffixes de verbes}

Cette étape trouve la racine d'un verbe.
On cherche et enlève les suffixes standards.

Faire cette étape seulement si l'étape 1 n'a rien changé ou si un des suffixes suivants a été trouvé: amment, emment, ment et ments.

\subsubsection*{Étape 2a: Suffixes débutant par i}

Trouver le suffixe le plus long parmi la liste ci-dessous.
S'il est précédé d'une consonne et que le tout est dans RV, supprimer le suffixe excluant la consonne supplémentaire.

îmes, ît, îtes, i, ie, ies, ir, ira, irai, iraIent etc.

\subsubsection*{Étape 2b: Autres suffixes de verbes}

Trouver le suffixe le plus long parmi les trois listes ci-dessous.
Les listes sont non-exaustives afin d'alléger le texte.

\begin{description}
  \item[ions]
  
  Supprimer si dans RV et R2;
  
  \item[é ée ées és èrent er era erai eraIent ...]
  
  Supprimer si dans RV;
  
  \item[âmes ât âtes a ai aIent ...]
  
  Supprimer si dans RV. Si la lettre e précède le suffixe et qu'elle se trouve aussi dans RV, la supprimer.
\end{description}

\subsection*{Étape 3: Dernière lettre}

Si les étapes 1 ou 2 ont modifié le mot, faire cette étape, sinon passer à l'étape 4.

Si la dernière lettre est un Y, la remplacer par i.
Si c'est un ç, la remplacer par c.

\subsection*{Étape 4: Résidue du suffixe}

Faire cette étape seulement si les étapes 1 à 3 n'ont pas modifié le mot.
Cette étape sert à enlever le pluriel et le féminin des mots proches de leurs racines.

Si un mot se termine par \textbf{s} et n'est pas précédé de \textbf{a, i, o, u, è ou s}, supprimer ce s final.

Si un mot se termine par \textbf{ion}, que cette finale se trouve dans RV et dans R2 et qu'elle est précédée de la lettre \textbf{s ou t} (cette dernière doit se trouver dans RV), supprimer la finale \textbf{ion}.

Si un mot se termine par \textbf{ier, ière, Ier ou Ière} et que cette finale se trouve dans RV, remplacer cette dernière par \textbf{i}.

Si un mot se termine par \textbf{e} et que cette lettre ce trouve dans RV, la supprimer.

Si un mot se termine par \textbf{ë}, que cette lettre se trouve dans RV et qu'elle est précédée de \textbf{qu}, la supprimer.

\subsection*{Étape 5: Dédoubler la lettre finale}

Les suppressions et remplacements des étapes précédentes peuvent laisser une faute dans la racine du mot.
Cette étape sert à enlever les lettres doubles de certaines finales de mots.

Si un mot se termine par \textbf{enn, onn, ett, ell ou eill}, supprimer la dernière lettre.

\subsection*{Étape 6: Accent final}

Cette étape aussi sert a nettoyer la racine du mot.

Si un mot se termine par la lettre \textbf{è ou é} suivit d'une ou de plusieurs consonnes, enlever l'accent de ce \textbf{e}.

\subsection*{Finalisation}

Finalement, on enlève les majuscules sur les voyelles ajoutées durant l'étape de préparation.

Au final, on obtient la racine du mot.
Lors de la comparaison de texte, il peut y avoir quelques problèmes.
Par exemple les mots \og acceptables \fg{} et \og accepté \fg{} ont la même racine: \og accept \fg{}.
Par contre le sens des 2 mots est différent.