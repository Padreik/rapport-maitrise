\chapter{Développement du module d'extension \og qtype\_essayhelper \fg{} }

Le développement d'un module d'extension Moodle n'est pas intuitif.
La documentation n'est pas toujours complète et l'interaction dans le code entre les types de modules d'extensions \og Rapport de questionnaire \fg{}, \og Type de question \fg{} et \og Comportement de question \fg{} n'est pas toujours claire.
Dans le développement de ce module d'extension, le code du module d'extension \og qtype\_essay \fg{} a été utilisé comme documentation.
En prenant ce module d'extension comme base, il a été facile de construire ce qui était désiré.

Les commentaires de copyrights ont été ajustés comme illustré à l'exemple de code~\ref{code:commentaire}

\begin{lstfloat}
\begin{lstlisting}[frame=l]
/**
 * Essay for correction helper question definition class.
 *
 * @package    qtype
 * @subpackage essayhelper
 * @copyright  2017 Philippe Girard
 * @license    http://www.gnu.org/copyleft/gpl.html GNU GPL v3 or later
 *
 * Inspired by:
 * @package    qtype
 * @subpackage essay
 * @copyright  2009 The Open University
 * @license    http://www.gnu.org/copyleft/gpl.html GNU GPL v3 or later
 */
\end{lstlisting}
\caption{Exemple des commentaires dans les fichiers du module d\'extension.}
\label{code:commentaire}
\end{lstfloat}

\section{Fonctionnalités}

Plusieurs fonctionnalités du module d'extension \og qtype\_essay \fg{} ont été enlevées:

\begin{description}
  \item[Éditeur de texte WYSIWIG]
  
  Il permet de modifier l'apparence du texte facilement.
  Ça permet, entre autres, de surligner, de souligner, de mettre en gras, de mettre en italique et plus encore.
  Comme on ne peut pas enlever des options WYSIWIG pour un seul module d'extension, ça devient complexe de trouver une mise en forme qui permettra de mettre en évidence les mots clés, car un étudiant pourrait, par erreur, reproduire la même mise en forme.
  
  \item[Remise de fichier]
  
  Il est possible d'écrire directement dans la zone de texte ou de remettre un document texte (configurable par l'enseignant).
  Comme le module d'extension n'aidera pas à corriger les textes remis par fichier, cette option a été enlevée.
\end{description}

Plusieurs fonctionnalités sont restées dans le nouveau module d'extension:

\begin{description}
  \item[Rétroaction générale]
  
  Permet de laisser un commentaire à l'étudiant une fois que la correction est disponible.
  Par exemple, l'enseignant pourrait laisser des explications pour les erreurs courantes.
  
  \item[Modèle de réponse]
  
  Prérempli la zone de texte de l'étudiant avec le texte donné.
  Par exemple, un en-tête de fonction pour une question de programmation ou la liste des mots à définir.
  
  \item[Information de l'évaluateur]
  
  Affiche un texte pour le correcteur seulement.
  Utile pour voir facilement le barème de correction ou donner des instructions au correcteur.
  Est affiché sous la réponse de l'étudiant lors de la correction.
\end{description}

Finalement, les fonctionnalités suivantes ont été ajoutées:

\begin{description}
  \item[Mots-clés]
  
  Les mots-clés seront mis en évidence dans la réponse de l'étudiant lors de la correction manuelle

  \item[Réponse officielle de l'enseignant]
  
  Affiche ce texte à droite de la réponse de l'étudiant lors de la correction.
  Les mots-clés seront mis en évidence aussi dans ce texte.
\end{description}

\section{Détection des mots clés}

Un module d'extension Moodle est programmée avec le langage PHP.
Le module d'extension développé devrait pouvoir fonctionner avec plusieurs langues afin de pouvoir le déployer sur le répertoire de modules d'extensions Moodle.

Les mots-clés peuvent avoir des différences d'accord ou de conjugaison dans le texte de l'étudiant lorsqu'on les compare avec les mots-clés fournis par l'enseignant.
Il faut donc ramener les mots à leur forme la plus simple.
Deux techniques existent, la lemmatisation et la racination (\og stemming \fg{} en anglais).

\begin{description}
  \item[Lemmatisation]
  
  La lemmatisation met le mot dans sa forme la plus simple (singulier, masculin, infinitif, etc.).
  Par exemple le verbe \og aimerait \fg{} sera transformé en \og aimer \fg{}.
  
  \item[Racination]
  
  La racination enlève la fin du mot afin d'en conserver seulement la racine.
  Par exemple le verbe \og aimerait \fg{} sera transformé en \og aim \fg{}.
\end{description}

La lemmatisation est une solution plus exacte que la racination, mais beaucoup plus complexe.
Mes recherches n'ont identifié aucune librairie PHP de lemmatisation fonctionnant avec plusieurs langues.
Par contre il existe une librairie de racination libre de droits appelés php-stemmer \cite{phpstemmer}.
Elle est sous \href{https://raw.githubusercontent.com/wamania/php-stemmer/master/LICENSE}{licence MIT} et utilise un algorithme développé par Martin Porter écrite dans un langage appelé Snowball \cite{snowball}.
php-stemmer permet de faire la racination des mots en français, anglais, espagnol, allemand, italien, russe et plusieurs autres.

Pour trouver la racine de tous les mots du texte, les caractères non alphanumériques sont remplacés par des espaces et le texte est découpé par les caractères d'espacements (espace, saut de ligne, tabulation, etc.).

\begin{lstfloat}
\begin{lstlisting}[frame=l]
$words = preg_split('/(\s|\')/', preg_replace('/[^[:alnum:][:space:]]/u', ' ', $sentence));
\end{lstlisting}
\caption{Isoler les mots du texte.}
\label{code:commentaire}
\end{lstfloat}

Ensuite, chaque mot est associé avec sa racine trouvée avec l'algorithme Snowball.
Chaque mot-clé a, préalablement, aussi été réduit à leur racine avec l'algorithme Snowball.

\begin{lstfloat}
\begin{lstlisting}[frame=l]
foreach ($words as $word) {
	if ($word) {
		if (Wamania\Snowball\Utf8::check($word)) {
			$stem = $stemmer->stem($word);
			if (isset($stems[$stem])) {
				if (!in_array($word, $stems[$stem])) {
					$stems[$stem][] = $word;
				}
			} else {
				$stems[$stem] = array($word);
			}
		} else {
			$stems[] = $word;
		}
	}
}
\end{lstlisting}
\caption{Racination des mots avec Snowball.}
\label{code:racinationsnowball}
\end{lstfloat}

Finalement les mots-clés trouvés dans le texte sont mis en évidence.

\begin{lstfloat}
\begin{lstlisting}[frame=l]
$usedKeywords = array_intersect(array_keys($stems), $keywords);

foreach ($usedKeywords as $keyword) {
	$words = $answerWords[$keyword];
	foreach ($words as $word) {
		$studentAnswer = str_replace($word, '<b><u>' . $word . '</u></b>', $studentAnswer);
	}
}
\end{lstlisting}
\caption{Mise en évidence des mots-clés trouvés.}
\label{code:commentaire}
\end{lstfloat}

\section{Tests d'acceptation}

TODO: Tests effectués dans mon module d'extension...

\section{Tests unitaires}

TODO: Tests effectués dans mon module d'extension...