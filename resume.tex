\begin{abstract}

Ce projet consiste à développer un module d'extension (\textit{plugin}) Moodle pour aider la correction des textes à développement dans une activité de type \og questionnaire \fg{}.

Le module d'extension développé est de type \og Type de question \fg{} basé sur le module d'extension \og \textit{qtype\_essay} \fg{}.
Le nouveau type de question se comporte  comme le type \og \textit{qtype\_essay} \fg{} hormis qu'il ne permet pas la remise de fichiers ou l'utilisation du \textit{WYSIWYG (What You See Is What You Get)}.
Plus sp\'ecifiquement, ce nouveau module ajoute les fonctionnalités suivantes au module de correction manuelle: mise en évidence de mots-clés trouvés par racination (\textit{stemming}) et affichage de la réponse de l'enseignant adjacent à la réponse de l'étudiant.

Des tests unitaires ainsi que des tests d'acceptation ont été écrits afin de valider le fonctionnement de ce module d'extension.

Les fonctionnalités de ce module d'extension sont limit\'ees. 
Notamment, des ajouts tels que les suivants seraient utiles pour apporter une aide plus importante au correcteur: remplacer la racination pour la lemmatisation afin d'améliorer l'identification des mots-clés~; comparer les textes des étudiants entre eux afin de détecter le plagiat~; effectuer une analyse syntaxique afin de donner une estimation de la note que mérite un texte basé sur les réponses précédemment corrigées.

MOTS-CLÉS: Moodle, Tests unitaires, Tests d'acceptation, Racination

\end{abstract}
